\documentclass[submission%
% if you want to use pdftex and pdflatex doesn't do well for you,
% uncomment the following line
%,pdftex%
% if you have difficulties with hyperref uncomment the following line
%,nohyperref%
% if you have difficulties with fonts uncomment the following line
%,notimes%
]{dmtcs}

% DON'T LOAD ANY STYLES THAT CHANGE THE PAGE LAYOUT
% AND DON'T CHANGE THE PAGE LAYOUT BY HAND, EITHER.

\usepackage{standalone}
\usepackage{ntheorem}
\usepackage[latin1]{inputenc}
\usepackage{subfigure}
\usepackage{mathrsfs}
\usepackage{../brunnian}
\usepackage{tikz}
\usepackage{subfigure}
\usepackage{pgfplots}
\usepgfplotslibrary{colorbrewer}
\usetikzlibrary{decorations.markings,backgrounds,hobby}

\tikzset{->-/.style={decoration={ markings, mark=at position #1
      with {\arrow{>}}},postaction={decorate}}}

% graphicx is now loaded automatically no need to put this in here anymore.
%
%\usepackage{graphicx}
\newcommand{\KnotDiaClass}{\mathscr{K}}
\newcommand{\KnotDiaGF}{K}
\newcommand{\KnotDiaCard}{k}

% just comment this out if you don't have natbib, or if you don't want
% to use it
\usepackage[round]{natbib}

\newtheorem*{conjecture}{Conjecture}
\newtheorem{theorem}{Theorem}
\newtheorem*{untheorem}{Theorem}
\newtheorem{corollary}{Corollary}
\newtheorem{proposition}{Proposition}
\theorembodyfont{\upshape}
\newtheorem*{definition}{Definition}

\author{Harrison Chapman\addressmark{1}\thanks{Email: \email{hchapman@math.uga.edu}}}
\title[Asymptotic laws for knot diagrams]{Asymptotic laws for knot diagrams}
% put your affiliation here, not your full address. If you like to give
% away your email address, put it in the \thanks as above.
\address{\addressmark{1}University of Georgia, Athens, GA}
\keywords{knots, maps, enumeration, immersed curves, DNA topology}
% don't try to cheat here, we will check the dates!
\received{2015-11-16}
\revised{\today}
\accepted{tomorrow}
\begin{document}
\maketitle
\begin{abstract}
  \paragraph{Abstract.}
  We study random knotting by considering knot and link diagrams as
  decorated, (rooted) topological maps on spheres and sampling them
  with the counting measure on from sets of a fixed number of vertices
  $n$. We prove that random rooted knot diagrams are highly composite
  and hence almost surely knotted (this is the analogue of the
  Frisch-Wasserman-Delbr\"uck conjecture) and extend this to unrooted
  knot diagrams by showing that almost all knot diagrams are
  asymmetric. The model is similar to one of Dunfield, \textit{et al.}

  \paragraph{R\'esum\'e.} Nous \'etudions les noeuds al\'eatoires en
  consid\'erant les diagrammes de noeuds comme cartes planaires
  d\'ecor\'ees sur lesquelles nous mettons la mesure uniforme de
  l'espace des cartes de $n$ sommets ($n$ fix\'e). Nous prouvons que
  les diagrammes de noeuds enracin\'es al\'eatoires sont tr\`es
  ``compos\'es'' et de fait presque s\^{u}rement nou\'es (ceci est un
  \'equivalent de la conjecture de Frisch-Wasserman-Delbr\"uck). Nous
  \'etendons ensuite  ce r\'esultat aux diagrammes de noeuds
  non enracin\'es en montrant que tous les diagrammes de noeuds sont
  asym\'etriques. Ce mod\`ele est similaire \`a celui de Dunfield,
  \textit{et al.}
\end{abstract}

\section{Introduction}
\label{sec:in}

\subsection{Definitions and motivations}
\label{sec:defin-motiv}

A \emph{planar map} (or simply \emph{map}) is a graph embedded in the
sphere $S^2$ up to continuous deformation. It is \emph{4-valent} if
the underlying graph is (\textit{i.e.}\ all vertices are of degree 4); we will
call 4-valent maps \emph{link shadows}. Vertices of link shadows will
be called \emph{crossings}. A map is \emph{decorated} by some set if
there is a mapping from the vertices of the map to the set. A
\emph{link component} of a link shadow is an equivalence class of
edges meeting opposite across crossings. If a link shadow consists of
precisely one link component, it is called a \emph{knot shadow}.

\begin{figure}[htbp]
  \centering
  \input{../unrooted_planar_map.brun}
  \caption{A planar map (left) and a planar map in the class of knot shadows (right).}
  \label{fig:planar-map-eg}
\end{figure}

The motivation for these definitions is thus: A \emph{knot type} is an
embedding of the circle $S^1$ into the 3-sphere $S^3$ modulo ``ambient
isotopy.'' Call a knot (resp.\ link) shadow decorated with $\{+,-\}$ a
\emph{knot (link) diagram}; this sign information is represented
graphically as ``over-under'' information (Figure~\ref{fig:overunder}).
\begin{figure}[htbp]
  \centering
  \subfigure[A figure-eight diagram viewed as a decorated shadow]{\input{../figure-8-big-dshad.brun}}
  \hfil
  \subfigure[A figure-eight diagram viewed as a knot drawing]{\input{../figure-8-big.brun}}
  \caption{After choosing once and for all a way to view signs as
    ``over-under'' information (\textit{i.e.}\ orientation around the
    knot), knot diagrams can be drawn as usual.}
  \label{fig:overunder}
\end{figure}
Equivalently then, a knot type is an equivalence class of knot
diagrams modulo the Reidemeister moves (Figure
\ref{fig:reidemeister}). The \emph{unknot} is the unique knot type
represented by the trivial loop in $S^3$. The unknot is
\emph{trivial}, and diagrams representing another knot type are
\emph{knotted}.

\begin{figure}[htbp]
  \centering
  \begin{tabular}{c@{\hspace{4em}}c@{\hspace{4em}}c}
  \centering
    \input{../reidemeister_1.brun}
    & \input{../reidemeister_2.brun}
    & \input{../reidemeister_3.brun}
  \end{tabular}
  \caption{The three Reidemeister moves}
  \label{fig:reidemeister}
\end{figure}

The \emph{random diagram model} of random knotting is then: Given a
number $n$ of crossings, sample uniformly a knot diagram with $n$
crossings and return its knot type. It is similar to models
of~\cite{diaoernst2012pnmkt} and~\cite{Dunfield:mdWrGjny}, but these
models do not sample from any well-understood measure on spaces of
knot diagrams. On the other hand, knot shadows are precisely curve
immersions on the sphere studied by~\cite{0036-0279-50-1-R01} and
rooted knot shadows are the open immersed plane curves
of~\cite{pzjschaeff2004planecurveasymp}.

In the context of DNA topology, \cite{Frisch_1961} and
~\cite{delbruck1962ams} independently conjectured;
\begin{conjecture}[Frisch-Wasserman-Delbr\"uck]
  As the size $n$ of a randomly sampled knot grows large, the
  probability that it is knotted tends to 1.
\end{conjecture}
The first proof of the conjecture was for $n$-step self-avoiding
lattice polygons, by~\cite{Sumners_1988}:
\begin{untheorem}[Sumners-Whittington]
  As the number of steps $n$ of a self-avoiding lattice polygon grows
  large, the probability that the polygon is knotted tends to 1
  exponentially quickly.
\end{untheorem}
Shortly thereafter the conjecture was proved in view of other models
of space curves; self-avoiding Gaussian polygons
(\cite{douglas94gauss}), self-avoiding equilateral polygons
(\cite{diao95eqpoly}), \textit{etc.} There are other (combinatorially
and algebraically interesting) models of random knotting, such as the
\emph{Petaluma} model of~\cite{petaluma1} or those which use elements
of the braid group (\textit{e.g.}\ in~\cite{Nechaev:1996gv}), but these are not
immediately geometrically motivated like the diagram model or random
space curve models.

Indeed, the primary purpose of this work is to ascertain that the
conjecture holds in our model;
\begin{theorem}
  \label{thm:knotted}
  As the number of crossings $n$ of a randomly sampled knot diagram
  grows large, the probability that the diagram is knotted tends to 1
  exponentially quickly.
\end{theorem}
This result will follow from results on the structure of knot
diagrams. However, as maps with symmetry are comparatively not well
understood, we consider objects with broken symmetry: A map together
with a choice of edge and a choice of direction is called
\emph{rooted} and considered up to the group of automorphisms which
preserve the root (\textit{i.e.}\ the directed edge)---the trivial
group. We will prove results for rooted knot diagrams, which will then
extend (asymptotically) to the unrooted case then after proving,
\begin{figure}[htbp]
  \centering
  \input{../rooted_planar_map.brun}
  \caption{A rooted planar map (left) and a rooted knot shadow (right).}
  \label{fig:planar-map-eg}
\end{figure}

\begin{theorem}
  \label{thm:asymm}
  As the number of crossings $n$ of an (unrooted) knot diagram grows
  large, the probability that the diagram has a nontrivial
  automorphism group tends to 0 exponentially quickly.
\end{theorem}

These two results answer two experimentally motivated questions posed
in~\cite{CCMknotdiagrams2015} in the affirmative. Indeed,
Theorem~\ref{thm:asymm} suggests that, for large $n$, experiments
(\textit{c.f.}\ Section~\ref{sec:sampling}) for unrooted knot diagrams
can be run instead on rooted knot diagrams and results will not differ
greatly. While sampling rooted knot diagrams uniformly is still
nontrivial, it is reasonably quick to generate rooted knot diagrams of
70 crossings (but nearly impossible to tabulate even all 12-crossing
unrooted diagrams).

We will by denote $\KnotDiaClass$ the class of rooted knot diagrams,
counted by the number of crossings. Unless mentioned otherwise, going
onward diagrams and shadows will be assumed rooted. We note that while
the counts of rooted link shadows (and hence rooted link diagrams) are
known precisely (they are in bijection with a class of blossom trees),
the \emph{knot condition} of having precisely one knot component has
made it difficult to count (or otherwise understand)
$\KnotDiaClass$. Indeed, there are
conjectures(\cite{pzjschaeff2004planecurveasymp}) and computer
computations
(\cite{zuber2015mapsimsperms,pzjtransfermtx,pzjmtxintchap}) of the
size, but there is as of yet no closed formula for the sizes
$\KnotDiaCard_n$ of $\KnotDiaClass_n$, the set of knot diagrams with
$n$ crossings. That an innocuous condition like ``only having one link
component'' makes counting the class of maps difficult is not unusual;
indeed, for meanders as well there are still only conjectures on the
asymptotics (\cite{DiFrancesco2000699}). We note that these conjectures
for rooted knot shadows and meanders are both made using intuition
from $2d$ quantum gravity.

Both Theorem~\ref{thm:knotted} and Theorem~\ref{thm:asymm} are proved
using a ``pattern theorem,'' which states that knot diagrams almost
surely contain appropriate substructure linearly often. Indeed,
Sumners and Whittington's original proof of the
Frisch-Wasserman-Delbr\"uck conjecture for self-avoiding lattice
polygons makes use of Kesten's pattern theorem for substructure in
self avoiding walks. In the case of knot diagrams (or knot shadows),
the appropriate substructure is that of \emph{tangle diagrams
  (shadows)}: A $2k$-tangle shadow with $n$ crossings is a map with
$n$ 4-valent vertices (crossings) and one \emph{boundary} vertex of
degree $2k$. A $2k$-tangle diagram is again a tangle shadow where all
(non-boundary) vertices are decorated with sign $\{+,-\}$. An example
tangle diagram can be seen in Figure~\ref{fig:tangleex}
\begin{figure}[hbtp]
  \centering
  \subfigure{\input{../6-tangle-bdyvert.brun}}\hfil
  \subfigure{\input{../6-tangle.brun}}
  \caption{A 6-tangle diagram with boundary vertex (left) and boundary
    vertex viewed as disk boundary (right).}
  \label{fig:tangleex}
\end{figure}
A tangle
diagram (resp.\ shadow) is contained in a link diagram (shadow) if
there is some generic open disk on the sphere in which the diagram
(shadow) is embedded that is map isomorphic (as maps on open disks) to
the interior of the tangle diagram (shadow). The pattern theorem is
then;

\begin{theorem}
  \label{thm:pattern}
  Fix a constant $c$. Let $P$ be a tangle diagram which can be
  appropriately attached to knot diagrams in $\KnotDiaClass$ (namely,
  attachment produces only knot diagrams). Let $h_n$ be the counts of
  diagrams in $\KnotDiaClass$ who are \emph{$P$-deficient}; they
  contain fewer than $cn$ pairwise disjoint copies of $P$. For $0 < c
  < 1$ sufficiently small, there exists $0 < d < 1$ so that,
  \begin{displaymath}
    \frac{h_n}{\KnotDiaCard_n} < d^n,
  \end{displaymath}
  \textit{i.e.}\ the diagrams in $\KnotDiaClass$ which do not contain at least
  $cn$ copies of $P$ are asymptotically exponentially rare.
\end{theorem}

The above theorem is proven first for rooted knot diagrams; the
asymmetry result of~\ref{thm:asymm} extends it immediately to unrooted
knot diagrams as well. We will see by our attachment constructions
that the pattern theorem is stronger than the bare result on
unknotting (Theorem~\ref{thm:knotted}): Not only is every knot diagram
(rooted or unrooted) knotted, it is a large, highly composite knot
diagram of many factors.

\subsection{Acknowledgements}
\label{sec:acknowledgements}

The author is extremely grateful to his advisor Jason Cantarella, for
his support, advice, and for introducing him to the knot tabulation project
(alongside Matt Mastin) and suggesting he prove
Theorem~\ref{thm:knotted}. The author is also grateful to the summer
school on applied combinatorics at the University of Saskatchewan and
CanaDAM thereafter, where he met and had meaningful conversations with
, and many others. The author also is indebted to funding from the NSF
(grant DMS-1344994 of the RTG in Algebra, Algebraic Geometry, and
Number Theory, at the University of Georgia), PIMS, the Simons Center,
and the AMS with which he was able to introduce his work to
others. The author is grateful for conversations with Julien Courtiel,
Elizabeth Denne, Chris Duffy, Chaim Even-Zohar, \'Eric Fusy, Gary
Iliev, Rafa\l{} Komendarczyk, Neal Madras, Kenneth Millett, Marni
Mishna, Erik Panzer, Jason Parsley, Eric Rawdon, Andrew Rechnitzer,
Clayton Shonkwiler, Chris Soteros, Karen Yeats, and many
others. Lastly, the author is grateful to Jean Gutt for his assistance
in translating the abstract into French.

\section{Results}

\subsection{The pattern theorem}
\label{sec:pattern}

The attachment defined in the pattern theorem~\ref{thm:pattern} is
rather technical. The pattern theorem itself is strongly based on
results of~\cite{Bender1992104} and the definition of attachment is
nearly the same;

\begin{definition}
  Let $P$ be a tangle diagram, and let $\mathscr{H}_n$ be the sets of
  knot diagrams in $\KnotDiaClass$ who contain fewer than $cn$ copies
  of $P$ as a subtangle. The tangle diagram $P$ can be
  \emph{appropriately attached} to knot diagrams in $\KnotDiaClass$ if
  there exists a (possibly larger) tangle diagram $Q$ containing $P$
  which can be attached to each $n$-crossing diagram $K$ in
  $\mathscr{H}$ in such a way that,
  \begin{enumerate}
  \item for some fixed, positive integer $k$, at least
    $\lfloor n/k \rfloor$ possible non-conflicting places of
    attachment exist,
  \item only diagrams in $\KnotDiaClass$ are produced,
  \item for any diagram produced as such we can identify the copies of
    $Q$ that have been added and they are all pairwise disjoint, and
  \item given the copies that have been added, the original diagram
    and the associated places of attachment are uniquely determined.
  \end{enumerate}
\end{definition}
The key piece then is the proposition,

\begin{proposition}
  Fix a constant $c$. Let $P$ be a tangle diagram which can be
  appropriately attached to knot diagrams in $\KnotDiaClass$. Let
  $\mathscr{H}$ the class of diagrams in $\KnotDiaClass$ who are
  $P$-deficient. Let the generating functions for
  $\KnotDiaClass$ and $\mathscr{H}$ be $\KnotDiaGF(z)$ and $H(z)$
  respectively. For $0 < c < 1$ sufficiently small,
  \begin{displaymath}
    r(\KnotDiaGF) < r(H),
  \end{displaymath}
  where $r(F)$ denotes the radius of convergence of $F$. By the
  Cauchy-Hadamard theorem, there is the equivalent result on the
  counting sequences;
  \begin{displaymath}
    \limsup_{n\to\infty}{\KnotDiaCard_n^{1/n}} > \limsup_{n\to\infty}{h_n^{1/n}}.
  \end{displaymath}
\end{proposition}

The proof of the proposition is nearly the same as that
in~\cite{Bender1992104}; it follows through without much alteration
for decorated maps. Appropriate attachment constructions however
require that the counts of link components remain fixed. The
proposition (and hence theorem~\ref{thm:pattern}) can be shown to
apply to subclasses of knot diagrams (and other classes of decorated
maps), but in the case of $\KnotDiaClass$ the construction is
illustrative and quite natural; it is analogous to the \emph{connect
  sum} operation on knots.

A $2$-tangle diagram is \emph{prime} if, after deleting the boundary
vertex, the tangle diagram's underlying graph is at least 4-edge
connected. Let $P =: Q$ be a prime $2$-tangle with at least $3$
crossings and exactly one link component. Then the following
attachment, ``connect sum,'' is admissible: Given a rooted diagram $K$
a (non-root) edge $e$, and choice of direction, the attachment of $Q$
into $K$ is given by deleting a (sufficiently small) open disk around
the midpoint of $e$ and replacing it by the tangle $Q$ and smoothing
edges appropriately (so there are no degree 2 vertices). It is a fact
that the knot type of a connect sum of a diagram and a tangle is
exactly the connect sum of the two knot types of the diagram and the
tangle (once closed and viewed as a diagram).
\begin{figure}[hbtp]
  \centering
  \subfigure[Diagram with marked edge]{\input{../figure-8-mkedge.brun}}
  \hfil
  \subfigure[A trefoil connect summand\label{fig:trefcs}]{\input{../trefoil_summand.brun}}
  \hfil
  \subfigure[After connect summation]{\input{../figure-8-cs-tref.brun}}
  \caption{Connect sum of a figure eight knot to a trefoil}
  \label{fig:csexample}
\end{figure}
An example of
this operation is given in Figure~\ref{fig:csexample}. Then,
\begin{enumerate}
\item There are $2n$ edges, so $2(2n-1)$ locations for
  attachment. With $k = 1$, we have that
  \begin{displaymath}
    2(2n-1) \ge \lfloor n \rfloor.
  \end{displaymath}
\item The result is indeed a diagram in $\KnotDiaClass$; connect sum
  attachment of $Q$ does not change the number of link components (the
  original diagram had one, as did the tangle $Q$; the two are
  identified after attachment and become one link component) and the
  map produced is exactly 4-valent.
\item Two instances of $Q$ must be pairwise disjoint in a knot diagram
  as $Q$ is prime.
\item Diagram connect sum is reversible (this is covered
  in~\cite{CCMknotdiagrams2015}), and so the original diagram and
  places of attachment can be recovered.
\end{enumerate}

Indeed, this construction together with the proposition do not
entirely prove Theorem~\ref{thm:pattern}. In fact, the last required
piece is that,

\begin{proposition}
  The limit $\lim_{n\to\infty}{\KnotDiaCard_n^{1/n}}$ exists and is
  equal to $\limsup_{n\to\infty}{\KnotDiaCard_n^{1/n}}$.
\end{proposition}

This proposition is of course clear in the case of link diagrams
(which are counted exactly), but not immediate in the case of knot
diagrams. The proof for the case of $\KnotDiaClass$ is almost entirely
that any two knot diagrams can be connect summed (similarly to the
above attachment construction) to produce a new knot diagram; some
care is required however as different pairs of diagrams can connect
sum to produce the \emph{same} knot diagram! It is hence useful to
apply the framework of~\cite{Bender1992104} for proving so-called
``smooth'' growth; appropriate constructions can apply to further
classes of knot diagrams as well.

Altogether, this proves Theorem~\ref{thm:pattern}. We can now show the
following corollary for \emph{rooted} knot diagrams (which is slightly
weaker than Theorem~\ref{thm:knotted}):

\begin{corollary}
  \label{thm:rootknotted}
  Let $u_n$ be the number of $n$-crossing \emph{rooted} knot diagrams which
  represent the unknot. Then there exists $0< d < 1$ so that,
  \begin{displaymath}
    \frac{u_n}{k_n} < d^n,
  \end{displaymath}
  \textit{i.e.} unknotted \emph{rooted} diagrams are exponentially
  rare among \emph{rooted} knot diagrams.
\end{corollary}

\begin{proof}
  Let $P$ be the $2$-tangle diagram in Figure~\ref{fig:trefcs},
  \textit{i.e.} a prime trefoil connect summand. Unknot diagrams are
  certainly $P$-deficient (otherwise they would represent a knot type
  which has trefoil factors), hence rare by Theorem~\ref{thm:pattern}.
  % \begin{figure}[htbp]
  %   \centering
  %   \input{../trefoil_summand.brun}
  %   \caption{A trefoil connect summand $2$-tangle diagram}
  %   \label{fig:trefcs}
  % \end{figure}
\end{proof}

\subsection{Asymmetry}
\label{sec:asymmetry}

It is a theorem of~\cite{Richmond19951} that ``almost all maps are
asymmetric,'' for maps which obey a pattern theorem, and indeed as we
shall see their theorem applies to knot shadows. Notice then that
almost all decorated knot shadows (\textit{e.g.}\ knot diagrams) are
asymmetric (they are arguably more asymmetric) since the decoration
imposes additional constraints. Notice, for example, that the connect
summation of a tangle to a diagram at a marked edge
(Figure~\ref{fig:csexample}) can occur in 4 different ways. Say a
tangle shadow $P$ is \emph{free} in a class of rooted knot shadows
$\mathscr{C}$ if any knot shadow obtained by removing a copy $P_1$ of
$P$ from a shadow $K$ and re-attaching $P$ in any fashion such that
the exterior legs of $P$ match up with the loose strands of $K
\setminus P_1$ is in $\mathscr{C}$.

We restate their theorem as it will apply to knot shadows;

\begin{untheorem}[\cite{Richmond19951}]
  Let $\mathscr{C}$ be a class of rooted knot shadows. Suppose that
  there is a tangle shadow $P$ such that for all knot shadows in
  $\mathscr{C}$, all copies of $P$ are pairwise disjoint and such that $P$
  \begin{enumerate}
  \item has no reflective symmetry in the plane,
  \item satisfies the hypotheses for the pattern
    Theorem~\ref{thm:pattern} for $\mathscr{C}$, and
  \item is free in $\mathscr{C}$.
  \end{enumerate}
  Then the proportion of $n$-crossing shadows in $\mathscr{C}$ with
  nontrivial automorphisms (that need not preserve the root) is
  exponentially small.
\end{untheorem}

Hence the proof of Theorem~\ref{thm:asymm} then depends only on the
appropriate construction.

\begin{proof}[of Theorem~\ref{thm:asymm}] The $2$-tangle shadow in
  figure~\ref{fig:asymtangle} satisfies the hypotheses at is a prime
  connect summand tangle shadow and is appropriately asymmetric.
  \begin{figure}[htbp]
    \centering
    \input{../asymmetric_2tangle_sh.brun}
    \caption{A $2$-tangle shadow satisfying the hypotheses for the
      asymmetry theorem (black) and its dual (gray).}
    \label{fig:asymtangle}
  \end{figure}
\end{proof}

This proves then the suppositions of
\cite{pzjschaeff2004planecurveasymp} and \cite{zuber2015mapsimsperms}.
It also follows that, asymptotically, there are simply $4n$ times as
many rooted knot shadows (or diagrams) as there are unrooted.

\subsection{Sampling}
\label{sec:sampling}

As we can obtain similar results for unrooted knot diagrams that we
can for \emph{rooted} knot diagrams, we can use the bijection
of~\cite{Schaeffer1997} between rooted 4-valent maps and blossom trees
to sample knot shadows efficiently. This yields the following uniform
sampling pipeline, which while imperfect, yields results for sizes ten
times the tabulation strategy of~\cite{CCMknotdiagrams2015}:

\begin{enumerate}
\item Sample a rooted 4-valent planar map with $n$ vertices
  (\textit{e.g.}, using PlanarMap of~\cite{SchaefferPlanarMap}). This can be
  done quickly, as the blossom trees are nearly the same as binary
  trees.
\item Reject until the map is a rooted knot shadow (\textit{i.e.} it
  has one link component). This becomes exponentially unlikely to
  succeed (because of the pattern theorem for link shadows).
\item For each of the $n$ crossings of the shadow, sample a sign from
  $\{+, -\}$, yielding a rooted knot diagram. This is trivial.
\end{enumerate}

An example of a random knot diagram produced as such appears in
Figure~\ref{fig:randomknot}. Notice that the embedding of the graphic
is independent of the actual knot diagram object.
\begin{figure}[htbp]
  \centering
  \definecolor{linkcolor0}{rgb}{0, 0, 0}
\begin{tikzpicture}[line width=.8, line cap=round, line join=round]
  \begin{scope}[color=linkcolor0]
    \draw (8.46, 6.83) -- (8.46, 6.92) -- (8.55, 6.92) -- (8.55, 6.83) -- (8.48, 6.83);
    \draw (8.44, 6.83) -- (8.39, 6.83);
    \draw (8.36, 6.83) -- (8.28, 6.83) -- (8.28, 7.01) -- (8.37, 7.01);
    \draw (8.37, 7.01) -- (8.64, 7.01);
    \draw (8.64, 7.01) -- (8.71, 7.01);
    \draw (8.75, 7.01) -- (8.82, 7.01) -- (8.82, 7.19) -- (8.73, 7.19);
    \draw (8.73, 7.19) -- (8.20, 7.19);
    \draw (8.07, 7.19) -- (7.93, 7.19) -- (7.93, 7.36);
    \draw (7.93, 7.36) -- (7.93, 7.45) -- (7.75, 7.45) -- (7.75, 7.36);
    \draw (7.75, 7.36) -- (7.75, 6.83);
    \draw (7.75, 6.83) -- (7.75, 0.95) -- (7.39, 0.95);
    \draw (7.21, 0.95) -- (2.13, 0.95) -- (2.13, 1.04) -- (7.21, 1.04);
    \draw (7.36, 1.04) -- (7.57, 1.04) -- (7.57, 1.18);
    \draw (7.57, 1.31) -- (7.57, 6.56);
    \draw (7.57, 6.67) -- (7.57, 6.74) -- (1.78, 6.74) -- (1.78, 6.67);
    \draw (1.78, 6.62) -- (1.78, 6.51);
    \draw (1.78, 6.44) -- (1.78, 6.33);
    \draw (1.78, 6.28) -- (1.78, 6.21) -- (1.69, 6.21) -- (1.69, 6.29) -- (1.78, 6.29);
    \draw (1.78, 6.29) -- (1.85, 6.29);
    \draw (1.88, 6.29) -- (1.94, 6.29);
    \draw (1.97, 6.29) -- (2.05, 6.29) -- (2.05, 6.47) -- (1.78, 6.47);
    \draw (1.78, 6.47) -- (1.42, 6.47);
    \draw (1.42, 6.47) -- (1.33, 6.47) -- (1.33, 7.54);
    \draw (1.33, 7.54) -- (1.33, 8.34);
    \draw (1.33, 8.47) -- (1.33, 8.61) -- (9.89, 8.61) -- (9.89, 0.15) -- (9.75, 0.15);
    \draw (9.69, 0.15) -- (9.62, 0.15) -- (9.62, 0.06) -- (9.71, 0.06) -- (9.71, 0.15);
    \draw (9.71, 0.15) -- (9.71, 8.34);
    \draw (9.71, 8.45) -- (9.71, 8.52) -- (9.80, 8.52) -- (9.80, 8.43) -- (9.71, 8.43);
    \draw (9.71, 8.43) -- (1.33, 8.43);
    \draw (1.33, 8.43) -- (0.71, 8.43);
    \draw (0.60, 8.43) -- (0.53, 8.43) -- (0.53, 8.52) -- (0.62, 8.52) -- (0.62, 8.43);
    \draw (0.62, 8.43) -- (0.62, 8.15);
    \draw (0.62, 8.04) -- (0.62, 7.93);
    \draw (0.62, 7.84) -- (0.62, 7.63) -- (0.98, 7.63) -- (0.98, 7.99) -- (0.80, 7.99) -- (0.80, 7.90);
    \draw (0.80, 7.90) -- (0.80, 7.81);
    \draw (0.80, 7.81) -- (0.80, 7.72) -- (0.71, 7.72) -- (0.71, 7.81) -- (0.78, 7.81);
    \draw (0.82, 7.81) -- (0.89, 7.81) -- (0.89, 7.90) -- (0.82, 7.90);
    \draw (0.76, 7.90) -- (0.62, 7.90);
    \draw (0.62, 7.90) -- (0.26, 7.90) -- (0.26, 8.70) -- (0.41, 8.70);
    \draw (0.46, 8.70) -- (0.53, 8.70) -- (0.53, 8.79) -- (0.44, 8.79) -- (0.44, 8.70);
    \draw (0.44, 8.70) -- (0.44, 8.17);
    \draw (0.44, 8.06) -- (0.44, 7.99) -- (0.35, 7.99) -- (0.35, 8.08) -- (0.44, 8.08);
    \draw (0.44, 8.08) -- (0.62, 8.08);
    \draw (0.62, 8.08) -- (1.07, 8.08);
    \draw (1.07, 8.08) -- (1.24, 8.08) -- (1.24, 8.25) -- (1.15, 8.25);
    \draw (1.15, 8.25) -- (1.07, 8.25);
    \draw (1.07, 8.25) -- (0.98, 8.25) -- (0.98, 8.17) -- (1.07, 8.17);
    \draw (1.07, 8.17) -- (1.15, 8.17) -- (1.15, 8.24);
    \draw (1.15, 8.27) -- (1.15, 8.34) -- (1.07, 8.34) -- (1.07, 8.27);
    \draw (1.07, 8.24) -- (1.07, 8.18);
    \draw (1.07, 8.15) -- (1.07, 8.09);
    \draw (1.07, 7.98) -- (1.07, 7.54);
    \draw (1.07, 7.54) -- (1.07, 0.69);
    \draw (1.07, 0.58) -- (1.07, 0.51) -- (1.69, 0.51) -- (1.69, 0.59);
    \draw (1.69, 0.59) -- (1.69, 0.74);
    \draw (1.69, 0.79) -- (1.69, 0.86) -- (1.60, 0.86) -- (1.60, 0.79);
    \draw (1.60, 0.76) -- (1.60, 0.68) -- (1.51, 0.68) -- (1.51, 0.77) -- (1.60, 0.77);
    \draw (1.60, 0.77) -- (1.69, 0.77);
    \draw (1.69, 0.77) -- (7.74, 0.77);
    \draw (7.87, 0.77) -- (8.02, 0.77);
    \draw (8.02, 0.77) -- (8.11, 0.77);
    \draw (8.11, 0.77) -- (8.19, 0.77) -- (8.19, 6.83) -- (8.12, 6.83);
    \draw (8.09, 6.83) -- (8.02, 6.83) -- (8.02, 0.87);
    \draw (8.02, 0.70) -- (8.02, 0.49);
    \draw (8.02, 0.40) -- (8.02, 0.33) -- (8.11, 0.33) -- (8.11, 0.42) -- (8.02, 0.42);
    \draw (8.02, 0.42) -- (7.93, 0.42) -- (7.93, 0.24) -- (9.35, 0.24) -- (9.35, 7.54);
    \draw (9.35, 7.54) -- (9.35, 7.72);
    \draw (9.35, 7.72) -- (9.35, 7.81) -- (9.26, 7.81) -- (9.26, 7.72) -- (9.34, 7.72);
    \draw (9.37, 7.72) -- (9.44, 7.72);
    \draw (9.44, 7.72) -- (9.53, 7.72) -- (9.53, 7.63) -- (9.44, 7.63) -- (9.44, 7.70);
    \draw (9.44, 7.74) -- (9.44, 7.81) -- (9.62, 7.81) -- (9.62, 7.54) -- (9.41, 7.54);
    \draw (9.26, 7.54) -- (1.43, 7.54);
    \draw (1.28, 7.54) -- (1.12, 7.54);
    \draw (1.05, 7.54) -- (0.98, 7.54) -- (0.98, 0.59) -- (1.07, 0.59);
    \draw (1.07, 0.59) -- (1.59, 0.59);
    \draw (1.78, 0.59) -- (7.84, 0.59) -- (7.84, 0.77);
    \draw (7.84, 0.77) -- (7.84, 6.83);
    \draw (7.84, 6.83) -- (7.84, 6.92) -- (7.93, 6.92) -- (7.93, 6.83) -- (7.86, 6.83);
    \draw (7.82, 6.83) -- (7.77, 6.83);
    \draw (7.65, 6.83) -- (1.78, 6.83);
    \draw (1.67, 6.83) -- (1.60, 6.83) -- (1.60, 6.92) -- (1.69, 6.92) -- (1.69, 6.83);
    \draw (1.69, 6.83) -- (1.69, 6.69);
    \draw (1.69, 6.63) -- (1.69, 6.56) -- (1.60, 6.56) -- (1.60, 6.65);
    \draw (1.60, 6.65) -- (1.60, 6.74) -- (1.51, 6.74) -- (1.51, 6.65) -- (1.58, 6.65);
    \draw (1.62, 6.65) -- (1.69, 6.65);
    \draw (1.69, 6.65) -- (1.78, 6.65);
    \draw (1.78, 6.65) -- (7.57, 6.65);
    \draw (7.57, 6.65) -- (7.66, 6.65) -- (7.66, 1.22) -- (7.57, 1.22);
    \draw (7.57, 1.22) -- (7.30, 1.22);
    \draw (7.30, 1.22) -- (2.23, 1.22);
    \draw (2.12, 1.22) -- (2.05, 1.22) -- (2.05, 1.13) -- (2.13, 1.13) -- (2.13, 1.22);
    \draw (2.13, 1.22) -- (2.13, 1.57);
    \draw (2.13, 1.76) -- (2.13, 6.12) -- (2.85, 6.12) -- (2.85, 5.94) -- (2.78, 5.94);
    \draw (2.72, 5.94) -- (2.58, 5.94);
    \draw (2.58, 5.94) -- (2.31, 5.94);
    \draw (2.31, 5.94) -- (2.22, 5.94) -- (2.22, 6.03) -- (2.31, 6.03) -- (2.31, 5.96);
    \draw (2.31, 5.84) -- (2.31, 1.66);
    \draw (2.31, 1.66) -- (2.31, 1.57) -- (2.85, 1.57) -- (2.85, 1.66);
    \draw (2.85, 1.66) -- (2.85, 3.71) -- (4.18, 3.71);
    \draw (4.36, 3.71) -- (4.72, 3.71) -- (4.72, 2.55) -- (4.63, 2.55);
    \draw (4.63, 2.55) -- (4.18, 2.55);
    \draw (4.18, 2.55) -- (4.10, 2.55) -- (4.10, 2.73) -- (4.18, 2.73);
    \draw (4.18, 2.73) -- (4.27, 2.73);
    \draw (4.27, 2.73) -- (4.54, 2.73);
    \draw (4.54, 2.73) -- (4.63, 2.73) -- (4.63, 2.59);
    \draw (4.63, 2.48) -- (4.63, 2.20) -- (4.18, 2.20);
    \draw (4.18, 2.20) -- (4.04, 2.20);
    \draw (3.99, 2.20) -- (3.92, 2.20) -- (3.92, 2.13);
    \draw (3.92, 2.06) -- (3.92, 1.90);
    \draw (3.92, 1.82) -- (3.92, 1.75) -- (3.03, 1.75) -- (3.03, 1.84);
    \draw (3.03, 1.84) -- (3.03, 3.09) -- (2.94, 3.09) -- (2.94, 1.84) -- (3.01, 1.84);
    \draw (3.12, 1.84) -- (3.92, 1.84);
    \draw (3.92, 1.84) -- (4.06, 1.84);
    \draw (4.11, 1.84) -- (4.17, 1.84);
    \draw (4.28, 1.84) -- (6.32, 1.84) -- (6.32, 1.91);
    \draw (6.32, 2.03) -- (6.32, 3.71);
    \draw (6.32, 3.90) -- (6.32, 4.60);
    \draw (6.32, 4.73) -- (6.32, 4.83);
    \draw (6.32, 4.89) -- (6.32, 4.96) -- (6.41, 4.96) -- (6.41, 4.87) -- (6.32, 4.87);
    \draw (6.32, 4.87) -- (5.97, 4.87);
    \draw (5.97, 4.87) -- (5.52, 4.87);
    \draw (5.52, 4.87) -- (5.34, 4.87) -- (5.34, 4.51);
    \draw (5.34, 4.51) -- (5.34, 4.42) -- (5.43, 4.42) -- (5.43, 4.51) -- (5.36, 4.51);
    \draw (5.33, 4.51) -- (5.25, 4.51);
    \draw (5.25, 4.51) -- (5.18, 4.51);
    \draw (5.09, 4.51) -- (4.81, 4.51);
    \draw (4.81, 4.51) -- (4.59, 4.51);
    \draw (4.52, 4.51) -- (4.45, 4.51) -- (4.45, 4.60) -- (4.54, 4.60) -- (4.54, 4.51);
    \draw (4.54, 4.51) -- (4.54, 4.07);
    \draw (4.54, 4.07) -- (4.54, 3.98) -- (4.45, 3.98) -- (4.45, 4.07) -- (4.52, 4.07);
    \draw (4.56, 4.07) -- (4.63, 4.07) -- (4.63, 3.89) -- (4.36, 3.89) -- (4.36, 4.69) -- (4.72, 4.69);
    \draw (4.83, 4.69) -- (4.90, 4.69);
    \draw (4.90, 4.69) -- (4.99, 4.69) -- (4.99, 4.78) -- (4.90, 4.78) -- (4.90, 4.71);
    \draw (4.90, 4.67) -- (4.90, 4.60) -- (5.08, 4.60) -- (5.08, 4.87) -- (4.81, 4.87) -- (4.81, 4.69);
    \draw (4.81, 4.69) -- (4.81, 4.55);
    \draw (4.81, 4.44) -- (4.81, 4.16);
    \draw (4.81, 4.16) -- (4.81, 4.09);
    \draw (4.81, 4.05) -- (4.81, 3.98) -- (5.52, 3.98) -- (5.52, 4.16);
    \draw (5.52, 4.16) -- (5.52, 4.30);
    \draw (5.52, 4.43) -- (5.52, 4.77);
    \draw (5.52, 4.89) -- (5.52, 4.96) -- (5.97, 4.96) -- (5.97, 4.89);
    \draw (5.97, 4.83) -- (5.97, 4.73);
    \draw (5.97, 4.67) -- (5.97, 4.60);
    \draw (5.97, 4.60) -- (5.97, 4.51) -- (5.88, 4.51) -- (5.88, 4.60) -- (5.95, 4.60);
    \draw (5.98, 4.60) -- (6.06, 4.60) -- (6.06, 4.42);
    \draw (6.06, 4.42) -- (6.06, 4.35);
    \draw (6.06, 4.32) -- (6.06, 4.25) -- (6.15, 4.25) -- (6.15, 4.34) -- (6.06, 4.34);
    \draw (6.06, 4.34) -- (5.52, 4.34);
    \draw (5.52, 4.34) -- (5.25, 4.34);
    \draw (5.25, 4.34) -- (5.16, 4.34) -- (5.16, 4.51);
    \draw (5.16, 4.51) -- (5.16, 4.95);
    \draw (5.16, 5.07) -- (5.16, 5.14) -- (5.25, 5.14) -- (5.25, 5.05);
    \draw (5.25, 5.05) -- (5.25, 4.61);
    \draw (5.25, 4.48) -- (5.25, 4.37);
    \draw (5.25, 4.30) -- (5.25, 4.16);
    \draw (5.25, 4.16) -- (5.25, 4.07) -- (4.81, 4.07);
    \draw (4.81, 4.07) -- (4.72, 4.07) -- (4.72, 4.16) -- (4.79, 4.16);
    \draw (4.90, 4.16) -- (5.16, 4.16);
    \draw (5.31, 4.16) -- (5.47, 4.16);
    \draw (5.62, 4.16) -- (6.23, 4.16) -- (6.23, 4.42) -- (6.09, 4.42);
    \draw (5.97, 4.42) -- (5.61, 4.42) -- (5.61, 4.78) -- (5.79, 4.78) -- (5.79, 4.71);
    \draw (5.79, 4.67) -- (5.79, 4.60) -- (5.70, 4.60) -- (5.70, 4.69) -- (5.79, 4.69);
    \draw (5.79, 4.69) -- (5.97, 4.69);
    \draw (5.97, 4.69) -- (6.32, 4.69);
    \draw (6.32, 4.69) -- (6.50, 4.69) -- (6.50, 5.05);
    \draw (6.50, 5.05) -- (6.50, 5.19);
    \draw (6.50, 5.28) -- (6.50, 5.49);
    \draw (6.50, 5.49) -- (6.50, 5.58) -- (5.34, 5.58) -- (5.34, 5.51);
    \draw (5.34, 5.42) -- (5.34, 5.14) -- (5.52, 5.14) -- (5.52, 5.21);
    \draw (5.52, 5.24) -- (5.52, 5.32) -- (5.43, 5.32) -- (5.43, 5.23) -- (5.52, 5.23);
    \draw (5.52, 5.23) -- (6.50, 5.23);
    \draw (6.50, 5.23) -- (6.79, 5.23);
    \draw (6.88, 5.23) -- (6.95, 5.23) -- (6.95, 5.32) -- (6.86, 5.32) -- (6.86, 5.23);
    \draw (6.86, 5.23) -- (6.86, 1.67);
    \draw (6.86, 1.56) -- (6.86, 1.49) -- (6.77, 1.49) -- (6.77, 1.57) -- (6.86, 1.57);
    \draw (6.86, 1.57) -- (7.04, 1.57);
    \draw (7.04, 1.57) -- (7.25, 1.57);
    \draw (7.32, 1.57) -- (7.39, 1.57) -- (7.39, 1.66) -- (7.30, 1.66) -- (7.30, 1.57);
    \draw (7.30, 1.57) -- (7.30, 1.40);
    \draw (7.30, 1.40) -- (7.30, 1.25);
    \draw (7.30, 1.18) -- (7.30, 1.04);
    \draw (7.30, 1.04) -- (7.30, 0.95);
    \draw (7.30, 0.95) -- (7.30, 0.86) -- (1.96, 0.86) -- (1.96, 1.66);
    \draw (1.96, 1.66) -- (1.96, 6.11);
    \draw (1.96, 6.22) -- (1.96, 6.29);
    \draw (1.96, 6.29) -- (1.96, 6.38) -- (1.87, 6.38) -- (1.87, 6.29);
    \draw (1.87, 6.29) -- (1.87, 6.21) -- (1.96, 6.21);
    \draw (1.96, 6.21) -- (2.94, 6.21);
    \draw (2.94, 6.21) -- (3.03, 6.21) -- (3.03, 6.12) -- (2.94, 6.12) -- (2.94, 6.19);
    \draw (2.94, 6.22) -- (2.94, 6.29) -- (3.11, 6.29) -- (3.11, 5.85) -- (2.76, 5.85);
    \draw (2.76, 5.85) -- (2.62, 5.85);
    \draw (2.54, 5.85) -- (2.40, 5.85) -- (2.40, 5.58) -- (2.67, 5.58) -- (2.67, 5.76) -- (2.60, 5.76);
    \draw (2.56, 5.76) -- (2.49, 5.76) -- (2.49, 5.67) -- (2.58, 5.67) -- (2.58, 5.76);
    \draw (2.58, 5.76) -- (2.58, 5.85);
    \draw (2.58, 5.85) -- (2.58, 5.92);
    \draw (2.58, 5.96) -- (2.58, 6.03) -- (2.76, 6.03) -- (2.76, 5.94);
    \draw (2.76, 5.94) -- (2.76, 5.87);
    \draw (2.76, 5.78) -- (2.76, 5.56);
    \draw (2.76, 5.40) -- (2.76, 3.71) -- (2.67, 3.71) -- (2.67, 5.49) -- (2.76, 5.49);
    \draw (2.76, 5.49) -- (3.04, 5.49);
    \draw (3.21, 5.49) -- (5.34, 5.49);
    \draw (5.34, 5.49) -- (6.41, 5.49);
    \draw (6.60, 5.49) -- (7.04, 5.49) -- (7.04, 1.67);
    \draw (7.04, 1.52) -- (7.04, 1.31) -- (7.21, 1.31) -- (7.21, 1.40);
    \draw (7.21, 1.40) -- (7.21, 1.49) -- (7.13, 1.49) -- (7.13, 1.40) -- (7.20, 1.40);
    \draw (7.23, 1.40) -- (7.29, 1.40);
    \draw (7.34, 1.40) -- (7.48, 1.40) -- (7.48, 5.67) -- (3.11, 5.67) -- (3.11, 5.49);
    \draw (3.11, 5.49) -- (3.11, 5.05) -- (4.10, 5.05);
    \draw (4.10, 5.05) -- (4.24, 5.05);
    \draw (4.37, 5.05) -- (5.16, 5.05);
    \draw (5.16, 5.05) -- (5.24, 5.05);
    \draw (5.35, 5.05) -- (6.41, 5.05);
    \draw (6.52, 5.05) -- (6.59, 5.05) -- (6.59, 2.11);
    \draw (6.59, 2.11) -- (6.59, 1.97);
    \draw (6.59, 1.91) -- (6.59, 1.84) -- (6.68, 1.84) -- (6.68, 1.93);
    \draw (6.68, 1.93) -- (6.68, 2.11) -- (6.61, 2.11);
    \draw (6.57, 2.11) -- (6.50, 2.11);
    \draw (6.50, 2.11) -- (6.41, 2.11) -- (6.41, 2.02) -- (6.50, 2.02) -- (6.50, 2.09);
    \draw (6.50, 2.20) -- (6.50, 3.80) -- (6.32, 3.80);
    \draw (6.32, 3.80) -- (4.27, 3.80);
    \draw (4.27, 3.80) -- (4.10, 3.80) -- (4.10, 4.95);
    \draw (4.10, 5.12) -- (4.10, 5.40) -- (5.16, 5.40) -- (5.16, 5.23) -- (4.27, 5.23);
    \draw (4.27, 5.23) -- (4.18, 5.23) -- (4.18, 5.32) -- (4.27, 5.32) -- (4.27, 5.24);
    \draw (4.27, 5.19) -- (4.27, 5.05);
    \draw (4.27, 5.05) -- (4.27, 3.90);
    \draw (4.27, 3.78) -- (4.27, 3.71);
    \draw (4.27, 3.71) -- (4.27, 3.01);
    \draw (4.27, 2.87) -- (4.27, 2.77);
    \draw (4.27, 2.71) -- (4.27, 2.64) -- (4.54, 2.64) -- (4.54, 2.71);
    \draw (4.54, 2.79) -- (4.54, 3.00) -- (4.36, 3.00) -- (4.36, 2.93);
    \draw (4.36, 2.89) -- (4.36, 2.82) -- (4.45, 2.82) -- (4.45, 2.91) -- (4.36, 2.91);
    \draw (4.36, 2.91) -- (4.27, 2.91);
    \draw (4.27, 2.91) -- (3.66, 2.91);
    \draw (3.54, 2.91) -- (3.47, 2.91) -- (3.47, 3.00) -- (3.56, 3.00);
    \draw (3.56, 3.00) -- (3.65, 3.00) -- (3.65, 3.09) -- (3.58, 3.09);
    \draw (3.47, 3.09) -- (3.11, 3.09) -- (3.11, 3.62) -- (3.65, 3.62) -- (3.65, 3.48);
    \draw (3.65, 3.43) -- (3.65, 3.36) -- (3.74, 3.36) -- (3.74, 3.44) -- (3.65, 3.44);
    \draw (3.65, 3.44) -- (3.58, 3.44);
    \draw (3.52, 3.44) -- (3.42, 3.44);
    \draw (3.35, 3.44) -- (3.20, 3.44) -- (3.20, 3.18) -- (3.47, 3.18) -- (3.47, 3.36) -- (3.40, 3.36);
    \draw (3.36, 3.36) -- (3.29, 3.36) -- (3.29, 3.27) -- (3.38, 3.27) -- (3.38, 3.36);
    \draw (3.38, 3.36) -- (3.38, 3.44);
    \draw (3.38, 3.44) -- (3.38, 3.53) -- (3.56, 3.53) -- (3.56, 3.44);
    \draw (3.56, 3.44) -- (3.56, 3.09);
    \draw (3.56, 3.09) -- (3.56, 3.02);
    \draw (3.56, 2.98) -- (3.56, 2.91);
    \draw (3.56, 2.91) -- (3.56, 2.82);
    \draw (3.56, 2.82) -- (3.56, 2.54);
    \draw (3.56, 2.43) -- (3.56, 2.32);
    \draw (3.56, 2.27) -- (3.56, 2.20) -- (3.83, 2.20) -- (3.83, 2.27);
    \draw (3.83, 2.34) -- (3.83, 2.55) -- (3.65, 2.55) -- (3.65, 2.47);
    \draw (3.65, 2.47) -- (3.65, 2.38) -- (3.74, 2.38) -- (3.74, 2.47) -- (3.67, 2.47);
    \draw (3.63, 2.47) -- (3.56, 2.47);
    \draw (3.56, 2.47) -- (3.47, 2.47) -- (3.47, 2.29) -- (3.56, 2.29);
    \draw (3.56, 2.29) -- (3.83, 2.29);
    \draw (3.83, 2.29) -- (3.97, 2.29);
    \draw (4.02, 2.29) -- (4.10, 2.29) -- (4.10, 2.38) -- (4.01, 2.38) -- (4.01, 2.29);
    \draw (4.01, 2.29) -- (4.01, 2.20);
    \draw (4.01, 2.20) -- (4.01, 2.13);
    \draw (4.01, 2.09) -- (4.01, 2.02) -- (4.10, 2.02) -- (4.10, 2.11) -- (4.01, 2.11);
    \draw (4.01, 2.11) -- (3.92, 2.11);
    \draw (3.92, 2.11) -- (3.38, 2.11) -- (3.38, 2.82) -- (3.52, 2.82);
    \draw (3.66, 2.82) -- (4.18, 2.82) -- (4.18, 2.75);
    \draw (4.18, 2.70) -- (4.18, 2.59);
    \draw (4.18, 2.48) -- (4.18, 2.27);
    \draw (4.18, 2.14) -- (4.18, 1.93);
    \draw (4.18, 1.93) -- (4.18, 1.84);
    \draw (4.18, 1.84) -- (4.18, 1.75) -- (4.10, 1.75) -- (4.10, 1.84);
    \draw (4.10, 1.84) -- (4.10, 1.93) -- (4.17, 1.93);
    \draw (4.28, 1.93) -- (6.32, 1.93);
    \draw (6.32, 1.93) -- (6.59, 1.93);
    \draw (6.59, 1.93) -- (6.66, 1.93);
    \draw (6.70, 1.93) -- (6.77, 1.93) -- (6.77, 1.66) -- (2.94, 1.66);
    \draw (2.75, 1.66) -- (2.41, 1.66);
    \draw (2.28, 1.66) -- (2.13, 1.66);
    \draw (2.13, 1.66) -- (1.99, 1.66);
    \draw (1.86, 1.66) -- (1.42, 1.66);
    \draw (1.42, 1.66) -- (1.33, 1.66) -- (1.33, 0.77) -- (1.42, 0.77) -- (1.42, 1.57);
    \draw (1.42, 1.76) -- (1.42, 6.38);
    \draw (1.42, 6.57) -- (1.42, 7.45) -- (1.87, 7.45) -- (1.87, 7.36);
    \draw (1.87, 7.36) -- (1.87, 6.92) -- (1.78, 6.92) -- (1.78, 7.36) -- (1.85, 7.36);
    \draw (1.96, 7.36) -- (7.65, 7.36);
    \draw (7.78, 7.36) -- (7.89, 7.36);
    \draw (8.02, 7.36) -- (9.09, 7.36);
    \draw (9.09, 7.36) -- (9.18, 7.36) -- (9.18, 7.27) -- (9.09, 7.27) -- (9.09, 7.35);
    \draw (9.09, 7.38) -- (9.09, 7.45) -- (9.26, 7.45) -- (9.26, 0.51) -- (8.11, 0.51) -- (8.11, 0.72);
    \draw (8.11, 0.87) -- (8.11, 6.83);
    \draw (8.11, 6.83) -- (8.11, 7.19);
    \draw (8.11, 7.19) -- (8.11, 7.27) -- (8.73, 7.27) -- (8.73, 7.20);
    \draw (8.73, 7.15) -- (8.73, 7.01);
    \draw (8.73, 7.01) -- (8.73, 0.77);
    \draw (8.73, 0.77) -- (8.73, 0.68) -- (8.64, 0.68) -- (8.64, 0.77);
    \draw (8.64, 0.77) -- (8.64, 6.91);
    \draw (8.64, 7.03) -- (8.64, 7.10) -- (8.37, 7.10) -- (8.37, 7.03);
    \draw (8.37, 6.97) -- (8.37, 6.83);
    \draw (8.37, 6.83) -- (8.37, 0.59) -- (9.09, 0.59) -- (9.09, 7.01) -- (8.91, 7.01) -- (8.91, 0.87);
    \draw (8.91, 0.76) -- (8.91, 0.68) -- (9.00, 0.68) -- (9.00, 0.77) -- (8.91, 0.77);
    \draw (8.91, 0.77) -- (8.77, 0.77);
    \draw (8.71, 0.77) -- (8.66, 0.77);
    \draw (8.60, 0.77) -- (8.46, 0.77) -- (8.46, 6.83);
  \end{scope}
\end{tikzpicture}

  \caption{A random knot diagram with 150 crossings}
  \label{fig:randomknot}
\end{figure}

The following experiment produced the data of
Figure~\ref{fig:knot-experiment}: For each $3 \le n \le 70$, sample
$20000$ knot diagrams using the above procedure, skipping a sample if
it does not produce a knot shadow after 150 attempts. Classify the
knot type (ignoring chirality) by calculating the HOMFLY-PT polynomial
of the diagram. The experiment took 3 hours to complete on an Intel
Core2 Duo 2.5 Ghz laptop, while it took nearly 7 days with an Amazon
Cloud Messaging set up (working in parallel on 20 or so machines) to
classify precisely all of the 10-crossing diagrams
in~\cite{CCMknotdiagrams2015}.

\begin{figure}[htbp]
  \centering
  \subfigure{
    \begin{tikzpicture}[scale=.8,
      every axis plot/.append style={line width=2pt}]
      \begin{axis}[
        %ymode=log,
        %ticks with fixed point,
        title={Number of successful samples},
        xlabel={\# crossings in knot diagram},
        ylabel={\# successful samples (out of 20,000)},
        %cycle list name=Set1-7,
        xmin=0, xmax=70,
        ymin=0, ymax=20100,
        mark repeat=10,
        ]
        \addplot[draw=black] table[x=n,y expr=\thisrow{total}] {sample.tsv};
        %\addplot table[x=n,y expr=\thisrow{3.1}/\thisrow{total}] {sample.tsv};
        %\addplot table[x=n,y expr=\thisrow{4.1}/\thisrow{total}] {sample.tsv};
        %\addplot table[x=n,y expr=\thisrow{5.1}/\thisrow{total}] {sample.tsv};
        %\addplot table[x=n,y expr=\thisrow{5.2}/\thisrow{total}] {sample.tsv};
        %\addplot table[x=n,y expr=\thisrow{3.1c3.1}/\thisrow{total}] {sample.tsv};
        %\addplot table[x=n,y expr=\thisrow{3.1c3.1m}/\thisrow{total}] {sample.tsv};
      \end{axis}
    \end{tikzpicture}}
  \subfigure{
    \begin{tikzpicture}[scale=.8,
      every axis plot/.append style={line width=2pt}]
      \begin{axis}[
        %ymode=log,
        %log ticks with fixed point,
        title={Ratios of knots in $n$-crossing diagrams},
        xlabel={\# crossings in diagram},
        ylabel={ratios of knot types to successful samples},
        %cycle list name=Set1-7,
        legend pos=north east,
        legend entries={$0_1$, $3_1$, $4_1$, $3_1\#3_1$},
        xmin=0, xmax=70,
        ymin=0, ymax=1,
        mark repeat=10,
        ]
        \addplot table[x=n,y expr=\thisrow{0.1}/\thisrow{total}] {sample.tsv};
        \addplot table[x=n,y expr=\thisrow{3.1}/\thisrow{total}] {sample.tsv};
        \addplot table[x=n,y expr=\thisrow{4.1}/\thisrow{total}] {sample.tsv};
        %\addplot table[x=n,y expr=\thisrow{5.1}/\thisrow{total}] {sample.tsv};
        %\addplot table[x=n,y expr=\thisrow{5.2}/\thisrow{total}] {sample.tsv};
        \addplot table[x=n,y expr=\thisrow{3.1c3.1}/\thisrow{total}] {sample.tsv};
        %\addplot table[x=n,y expr=\thisrow{3.1c3.1m}/\thisrow{total}] {sample.tsv};
      \end{axis}
    \end{tikzpicture}}
  \caption{Number of successful samples (left). Plot of various knot types recorded during sampling (right).}
  \label{fig:knot-experiment}
\end{figure}

Notably, the data shows (as expected) that the unknot ratio (indeed,
the ratio of all fixed knot types observed) tends to zero
exponentially quickly. Furthermore, there is a point at around 40
crossings where the ratio of trefoils surpasses that of
unknots. Indeed, we show in the full version~\cite{chapman2015} that
no one knot type is most common for all time.

\section{Conclusion}
\label{sec:conclusion}

We have covered some of the results for general knot shadows and knot
diagrams. However, the machinery permits more (provided appropriate
constructions). For example, we provide a proof in the full
version~\cite{chapman2015} of the pattern theorem (and beyond) for
subclasses of knot diagrams; reduced diagrams, who are at least
$2$-vertex connected, and prime diagrams, who are at least $4$-edge
connected. Respectively these two subclasses refer to diagrams without
``nugatory crossings'' whose crossing sign does not affect knot type
and diagrams who cannot be realized as a diagram connect sum of two
smaller diagrams, and are interesting topologically.

As well, the topological map language enables one to prove further
results for different kinds of knot diagrams. For example, by instead
considering knot shadows on orientable surfaces of arbitrary genus
(rather than simply the sphere), one arrives at the class of
\emph{virtual} knot diagrams. This is a different combinatorial way to
approach the subject than that of Gauss diagrams, which may be easier
to count.

As the Frisch-Wasserman-Delbr\"uck conjecture holds for the random
diagram model, it is hoped that the powerful tools that exist for maps
on surfaces can be applied to prove further results for the random
diagram model which are physically relevant.

\bibliographystyle{abbrvnat}
% use the following instead if you encounter problems
%\bibliographystyle{alpha}
\bibliography{../sources}
\label{sec:biblio}

\end{document}
