\documentclass[12pt]{article}
\usepackage{graphicx}
\usepackage{mathptm}
\usepackage{fancyheadings}
\topmargin 12mm
\advance \topmargin by -\headheight
\advance \topmargin by -2.5cm
\advance \topmargin by -\headsep
     
\textheight 9.9in
     
\oddsidemargin 0cm
\evensidemargin \oddsidemargin
\marginparwidth 0.5in
     
\setlength{\parskip}{12pt}
\textwidth 6.3in
% \parindent=0cm

\begin{document}
\pagestyle{empty}
%\cfoot{Recommendation letter CFG}
%\begin{flushright}
%\begin{minipage}{40mm}
%\includegraphics[width=40mm]{frlogocol.pdf}
%\end{minipage}\bigskip\\
%%\begin{minipage}{5cm}
%{\footnotesize Clayton Shonkwiler}\\
%{\scriptsize
%Department of Mathematics\\
%Fort Collins, Colorado 80523-1874\\
%Telephone: ++1(970)491-1822\\
%FAX: ++1(970)491-2161\\
%{\tt clayton@math.colostate.edu}\\
%\ \\
%\date\\}
%%\end{minipage}
%\end{flushright}
%\medskip

% \begin{minipage}{6.3in}
\noindent 
Dear Editors,

My coauthors and I would like to propose our paper ``Knot Probabilities in Random Diagrams'' for publication in J.\ Phys.\ A. 
The topic of random knots has been of interest to physicists for many years and has been well-covered in the pages of JPA.

In this paper, we present the results of a computational experiment enumerating the knot diagrams with 10 and fewer crossings (about 2 billion diagrams altogether), which yield exact probabilities for knots generated by randomly selecting a diagram. 
The unique feature of our dataset is that one can reliably compute the probabilities of all the knot types represented in the space, with knot frequencies ranging over 9 orders of magnitude. This is the first time that data with this much range has been presented in the random knotting literature, and the data set should be very useful for future experiments. 

Our data suggests a number of interesting conjectures about the relative frequencies of random knots, which we explore in the text. However, we view the dataset itself as the primary contribution of this experimental paper. We intend to make the entire dataset available as supplementary data if our paper is accepted. We also present a complete and formal proof that our algorithm for generating the dataset is valid, for those interested in the details of the experiment. 

We think that this combination of methods, together with the long history of the journal in random knotting, makes J.\ Phys.\ A.\ a natural home for this manuscript. 

\bigskip\bigskip

\qquad\qquad Best regards,

\vspace{-.1in}

%\qquad\qquad \includegraphics[scale=1]{ShonkwilerSignature.pdf}

\vspace{-.1in}

\qquad\qquad \begin{minipage}{4in}
Jason Cantarella\\
Professor, Mathematics\\
University of Georgia
\end{minipage}

P.S. We have relegated some of the longer proofs to appendices. We include them for the convenience of the referees, who may wish to check the details. However, these may be cut in the final version if the editors feel that they contain too much mathematical detail to interest the readership of the journal. 

% \end{minipage}
\end{document}
\end{minipage}
\end{document}
