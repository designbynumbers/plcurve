\documentclass[12pt]{article}
\usepackage{graphicx}
\usepackage{mathptm}
\usepackage{fancyheadings}
\topmargin 12mm
\advance \topmargin by -\headheight
\advance \topmargin by -2.5cm
\advance \topmargin by -\headsep
     
\textheight 9.9in
     
\oddsidemargin 0cm
\evensidemargin \oddsidemargin
\marginparwidth 0.5in
     
\setlength{\parskip}{12pt}
\textwidth 6.3in
% \parindent=0cm

\begin{document}
\pagestyle{empty}

REFEREE REPORT(S):
Referee: 1

COMMENTS TO THE AUTHOR(S) This is an interesting and worthwhile paper. The
authors examine the relative frequencies of occurrence of different knot types
in the world of knot diagrams, where diagrams with a fixed number of crossings
are assigned uniform measure. This is not the same as the more usually studied
model where different simple closed curves in 3-space are assigned uniform
measure. The model studied here is less physically relevant but none the less
interesting. The authors present and discuss an enumeration scheme and give
interesting results about relative frequencies. In addition the paper contains a
lot of oddments of information that are themselves of considerable interest. The
arguments given are clear and the paper is well written. I have only a few minor
comments that might somewhat improve the paper.

\begin{enumerate}
\item  On page 1, near the end of the first paragraph, I think that several
  pertinent references are missing. There is a paper by Soteros et al in Math
  Proc Camb Phil Soc in about 1992 and two papers by Diao in JKTR (about the
  same time) that treat various aspects of different types of simple closed
  curves in 3-space. I would like to see these references included.

  \emph{Added the sources
    \begin{itemize}
    \item
      http://journals.cambridge.org/action/displayAbstract?fromPage=online\&aid=2095592
    \item
      http://www.worldscientific.com.proxy-remote.galib.uga.edu/doi/abs/10.1142/S0218216595000090
    \item
      http://www.worldscientific.com.proxy-remote.galib.uga.edu/doi/abs/10.1142/S0218216594000307
    \end{itemize}
    to the end of paragraph 1.
  }

\item On page 2 the authors say that their results obey Zipf's law (or an
  approximation to it). This has been observed in other models. See a paper by
  Orlandini et al in Phys Rev Letters in 2007.

  \emph{Added the source
    http://journals.aps.org/prl/abstract/10.1103/PhysRevLett.99.058301 and a
    reference in the text: ``This distribution has been previously observed for
    knotting in self-avoiding polygon models by Baiesi, et
    al.~[\#].''}

\item On page 14, above Table I, is the detailed discussion of the organization
  necessary?

\item On page 15, just before Table II, the authors talk about symmetries
  becoming rare. Do they have information about the rate? Is it exponential?

  \emph{One of the authors has shown that the rate at which diagrams are
    asymmetric is at least exponential, so ``decrease'' has been changed to
    ``decrease exponentially''.}

\item In Table III, why doesn't \(5_1\) appear?

  \emph{The knot types \(5_1\) and \(5_1^m\) have been added to the table.}

\item On page 18, Zipf's law is mentioned again. See my earlier comment.

  \emph{See (2). Added a citation to the reference: ``(cf.~[\#])''}
\end{enumerate}

Referee: 2

COMMENTS TO THE AUTHOR(S) The manuscript is overall well-written: sufficient
details are provided to confirm the authors' propositions, lemmas, and theorems.
The presented analysis is detailed and thorough. The only criticism is that at
times the manuscript reads like a novel: commentaries extraneous to the
scientific matter should be eliminated; only the material and comments relating
to the academic comment (as opposed to the authors' thought processes) should be
included. This should reduce the length of the manuscript while making it more
focused.


\end{document}
\end{minipage}
\end{document}
