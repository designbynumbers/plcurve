\documentclass[amsmath,secnumarabic,floatfix,amssymb,nofootinbib,nobibnotes,letterpaper,11pt,tightenlines,showkeys]{revtex4}

\usepackage{times}

\usepackage{geometry}
\usepackage{amssymb}
\usepackage{latexsym, amsmath, amscd,amsthm}
\usepackage{graphicx}
\usepackage[percent]{overpic}
%\usepackage{pdfsync}
\usepackage{units}
\usepackage{hyperref}
\PassOptionsToPackage{caption=false}{subfig}
\usepackage[lofdepth]{subfig}

\usepackage{clrscode}

\def\figdir{figs/}
\graphicspath{\figdir}

\newtheorem{theorem}{Theorem}
\newtheorem*{maintheorem}{Main Theorem}
\newtheorem{lemma}[theorem]{Lemma}
\newtheorem{proposition}[theorem]{Proposition}
\newtheorem{corollary}[theorem]{Corollary}

\theoremstyle{definition}
\newtheorem{definition}[theorem]{Definition}
\newtheorem*{example}{Example}
\newtheorem{conjecture}[theorem]{Conjecture}
\newtheorem{remark}[theorem]{Remark}

\def\defn#1{Definition~\ref{def:#1}}
\def\thm#1{Theorem~\ref{thm:#1}}
\def\lem#1{Lemma~\ref{lem:#1}}
\def\figr#1{Figure~\ref{fig:#1}}
\def\prop#1{Proposition~\ref{prop:#1}}
\def\cor#1{Corollary~\ref{cor:#1}}
\def\sect#1{Section~\ref{sect:#1}}
\def\mainthm#1{Main Theorem~\ref{mainthm:#1}}
\def\mainthm#1{Main Theorem~\ref{mainthm:#1}}
\def\rmark#1{Remark~\ref{rmark:#1}}
%\numberwithin{equation}{section}

% make a small change

\newcommand{\abs}[1]{\lvert#1\rvert}
\newcommand{\tvnorm}[1]{\left| #1 \right|_{\operatorname{TV}}}
\newcommand{\R}{\mathbb{R}}
\newcommand{\C}{\mathbb{C}}
\newcommand{\Q}{\mathbb{H}}
\newcommand{\Z}{\mathbb{Z}}
\newcommand{\F}{\mathbb{F}}

\newcommand{\arc}[1]{\gamma_{#1}}
\newcommand{\len}[1]{\ell_{#1}}
\newcommand{\bdry}{\partial}
\newcommand{\bdy}{\bdry}
\newcommand{\isom}{\cong}
\newcommand{\setm}{\smallsetminus}
\newcommand{\eps}{\varepsilon}
\newcommand{\lk}{\textrm{lk}}
\newcommand{\intr}{\textrm{int}} %interior
\newcommand{\half}{\tfrac12}
\newcommand{\arcsec}{\textrm{arcsec}}
\newcommand{\m}{\mathcal}
\renewcommand{\d}{\partial}
\newcommand{\grad}{\nabla}
\newcommand{\PThree}{\ePol_3(n)/\SO(3)}

\bibliographystyle{plain}
%
\def\co{\colon\!}

\setlength{\parskip}{5pt}

\let\mgp=\marginpar \marginparwidth18mm \marginparsep1mm
\def\marginpar#1{\mgp{\raggedright\tiny #1}}
%\def\marginpar#1{}   %Uncomment this to hide all marginpars

\let\lbl=\label
\def\label#1{\lbl{#1}\ifinner\else\marginpar{\ref{#1} #1}\ignorespaces\fi}

\bibliographystyle{plain}

\begin{document}
\title[]{Knot Probabilities in Random Diagrams}
\author{Jason Cantarella, Harrison Chapman, Eric Lybrand, Hollis Neel and Malik Henry}
\altaffiliation{University of Georgia, Mathematics Department, Athens GA}
\noaffiliation
\author{Matt Mastin}
\altaffiliation{Wake Forest University, Mathematics Department, Athens GA}
\noaffiliation
\author{Eric Rawdon(?)}
\altaffiliation{Wake Forest University, Mathematics Department, Athens GA}
\noaffiliation

\maketitle

Suppose that one is given an $n$-crossing knot diagram chosen at random from the (finite) set of such diagrams. What is the probability that it is a diagram of the unknot? In this paper, we report on a computer experiment which gives precise answers to this and similar questions for $n \leq 12$ by direct enumeration and classification of knot diagrams. From the point of view of classical knot theory, this is a particularly simple model of random knotting. Part of our interest is to provide data which can be compared to results about more complicated distributions, such as the distribution of knots provided by selecting random closed equilateral $n$-gons, closed lattice walks, or in combinatorial models such as Even-Zohar et.\ al.\'s \emph{Petaluma} model.

\section{Definitions}

definition of diagram
arnold's plane curve invariants
equivalence relations

\section{Constructing the database of diagrams}

plantri and the plantri theorem
reduction theorem for shadows
reduction for diagrams

\section{Classifying knot types}

homfly
mathematica
snappy

\section{Results}

our distributions
monogon and bigon fractions
degree of alternatingness
universal properties? comparison with distribution from ERPs, lattice walks, and petaluma.

\section{Future Directions}


\end{document}
