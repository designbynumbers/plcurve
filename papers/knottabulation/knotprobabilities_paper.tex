\documentclass[amsmath,secnumarabic,floatfix,amssymb,nofootinbib,nobibnotes,letterpaper,11pt,tightenlines,showkeys]{revtex4}

\usepackage{times}

\usepackage{geometry}
\usepackage{amssymb}
\usepackage{latexsym, amsmath, amscd,amsthm}
\usepackage{graphicx}
\usepackage[percent]{overpic}
%\usepackage{pdfsync}
\usepackage{units}
\usepackage{hyperref}
\PassOptionsToPackage{caption=false}{subfig}
\usepackage[lofdepth]{subfig}

\usepackage{clrscode}

\def\figdir{figs/}
\graphicspath{\figdir}

\newtheorem{theorem}{Theorem}
\newtheorem*{maintheorem}{Main Theorem}
\newtheorem{lemma}[theorem]{Lemma}
\newtheorem{proposition}[theorem]{Proposition}
\newtheorem{corollary}[theorem]{Corollary}

\theoremstyle{definition}
\newtheorem{definition}[theorem]{Definition}
\newtheorem*{example}{Example}
\newtheorem{conjecture}[theorem]{Conjecture}
\newtheorem{remark}[theorem]{Remark}

\def\defn#1{Definition~\ref{def:#1}}
\def\thm#1{Theorem~\ref{thm:#1}}
\def\lem#1{Lemma~\ref{lem:#1}}
\def\figr#1{Figure~\ref{fig:#1}}
\def\prop#1{Proposition~\ref{prop:#1}}
\def\cor#1{Corollary~\ref{cor:#1}}
\def\sect#1{Section~\ref{sect:#1}}
\def\mainthm#1{Main Theorem~\ref{mainthm:#1}}
\def\mainthm#1{Main Theorem~\ref{mainthm:#1}}
\def\rmark#1{Remark~\ref{rmark:#1}}
%\numberwithin{equation}{section}

% make a small change

\newcommand{\abs}[1]{\lvert#1\rvert}
\newcommand{\tvnorm}[1]{\left| #1 \right|_{\operatorname{TV}}}
\newcommand{\R}{\mathbb{R}}
\newcommand{\C}{\mathbb{C}}
\newcommand{\Q}{\mathbb{H}}
\newcommand{\Z}{\mathbb{Z}}
\newcommand{\F}{\mathbb{F}}

\newcommand{\arc}[1]{\gamma_{#1}}
\newcommand{\len}[1]{\ell_{#1}}
\newcommand{\bdry}{\partial}
\newcommand{\bdy}{\bdry}
\newcommand{\isom}{\cong}
\newcommand{\setm}{\smallsetminus}
\newcommand{\eps}{\varepsilon}
\newcommand{\lk}{\textrm{lk}}
\newcommand{\intr}{\textrm{int}} %interior
\newcommand{\half}{\tfrac12}
\newcommand{\arcsec}{\textrm{arcsec}}
\newcommand{\m}{\mathcal}
\renewcommand{\d}{\partial}
\newcommand{\grad}{\nabla}
\newcommand{\PThree}{\ePol_3(n)/\SO(3)}
\newcommand{\EOne}{E_1}
\newcommand{\ETwo}{E_2}
\newcommand{\FOne}{F_1}

\bibliographystyle{plain}
%
\def\co{\colon\!}

\setlength{\parskip}{5pt}

\let\mgp=\marginpar \marginparwidth18mm \marginparsep1mm
\def\marginpar#1{\mgp{\raggedright\tiny #1}}
%\def\marginpar#1{}   %Uncomment this to hide all marginpars

\let\lbl=\label
\def\label#1{\lbl{#1}\ifinner\else\marginpar{\ref{#1} #1}\ignorespaces\fi}

\bibliographystyle{plain}

\begin{document}
\title[]{Knot Probabilities in Random Diagrams}
\author{Jason Cantarella, Harrison Chapman, Eric Lybrand, Hollis Neel and Malik Henry}
\altaffiliation{University of Georgia, Mathematics Department, Athens GA}
\noaffiliation
\author{Matt Mastin}
\altaffiliation{Wake Forest University, Mathematics Department, Athens GA}
\noaffiliation
\author{Eric Rawdon(?)}
\altaffiliation{Wake Forest University, Mathematics Department, Athens GA}
\noaffiliation

\maketitle

Suppose that one is given an $n$-crossing knot diagram chosen at random from the (finite) set of such diagrams. What is the probability that it is a diagram of the unknot? In this paper, we report on a computer experiment which gives precise answers to this and similar questions for $n \leq 12$ by direct enumeration and classification of knot diagrams. From the point of view of classical knot theory, this is a particularly simple model of random knotting. Part of our interest is to provide data which can be compared to results about more complicated distributions, such as the distribution of knots provided by selecting random closed equilateral $n$-gons, closed lattice walks, or in combinatorial models such as Even-Zohar et.\ al.\'s \emph{Petaluma} model.

\section{Definitions}

definition of diagram
arnold's plane curve invariants
equivalence relations

\section{Constructing the database of diagrams}

Brinkman and McKay~\cite{Brinkmann:2007,McKay:1998wa} gave an algorithm for producing isomorphism-free classes of embedded planar graphs, implemented in the software \texttt{plantri}. 


In the spirit of Brinkmann and McKay, we now define two expansion moves which change graph type and embedding and a modification of the planar embedding: 
\begin{definition}
The expansion move $\EOne$ (``loop creation'') adds a loop edge to an existing vertex. The expansion move $\ETwo$ (``bigon creation'') doubles an existing edge. We'll call the reverse of these moves ``reduction'' moves ``loop collapse'' and ``bigon collapse''.
\begin{figure}[h]
\begin{center}
Expansion moves graphic here
\end{center}
\end{figure}
\label{def:expansions}
We also define a move that we call $\FOne$ (``flip''). If a single pair of parallel edges joins two vertices, the resulting loop can separate a portion of the graph from another portion. This is a pair of parallel edges which does not correspond to a bigon face. In this case, we can flip one portion of the graph to the other side of the loop at one of the two crossings, as in the move below:
\begin{figure}[h]
\begin{center}
Flipping out graphic here
\end{center}
\end{figure}
\end{definition}

We can now show

\begin{proposition}
Every connected $4$-regular embedded planar (multi)graph $G$ can be obtained from a connected, embedded planar simple graph of vertex degree $\leq 4$ $G_0$ by a series of $\EOne$ and $\ETwo$ expansions. The graph isomorphism type of $G_0$ is canonical, but the embedding of $G_0$ depends on an arbitrary choice of exterior face in the embedding of $G$.
\end{proposition}

\begin{proof} 
We proceed by descent, using the reverse of the two expansion moves. Suppose we have a connected $4$-regular embedded planar multigraph $G$. Denote (arbitrarily) a single face of the embedding which is not a monogon face as the ``exterior'' face $f_0$. 

If $G$ is simple, there is nothing to prove. If not, $G$ has a loop edge, which can be removed by an $\EOne$ loop collapse, or a pair of edges joining the same endpoints. If this pair defines a bigon face which is not the distinguished face $f_0$, it can be removed by an $\ETwo$ bigon collapse. 

If the pair of edges does not bound a bigon face, or bounds only $f_0$ as a bigon face, we must perform a flip move. We do so with a little care to avoid the possibility of an infinite number of flips. The cycle generated by the pair of edges separates the sphere on which the diagram is embedded. One side is exterior (contains $f_0$) and the other is interior, and there is some portion of the graph contained in the interior. We can perform an $\FOne$ move at one (or both) vertices (as below) to flip whichever portion of the graph is interior to the outside. This creates a bigon face, which we can then collapse with an $\ETwo$ move.

\begin{figure}[h]
\begin{center}
Flipping out graphic here. 
\end{center}
\end{figure}

All of these three options reduce the number of edges by one, so there are finitely many steps.
When the process terminates, we have removed all loop edges and replaced all sets of parallel edges with single edges, yielding a simple graph $G_0$. The isomorphism type of $G_0$ is determined, since $\FOne$ does not change graph isomorphism type, and $\ETwo$ and $\EOne$ commute (with respect to graph isomorphism type).

We can maintain an embedding during the process to arrive at an embedding for the simple graph $G_0$. Since we don't destroy the exterior face during the descent process, the embedding has a marked exterior face, as well. 

\textbf{NOTE:} The question remains--- does a different choice of exterior face lead to the same embedding of the graph $G_0$ (with a corresponding different choice of exterior face)? Or can different choices of exterior face lead to different embeddings of the $G_0$ after the process is complete? Commutativity of the operations is the issue here. It is clear that loop edges must occur at different vertices by the vertex degree bound (excluding the trivial case of the figure-$8$ graph), so loop collapses commute with each other. Similarly, bigon collapses with disjoint vertex sets commute with each other, and with loop collapses as long as the loop isn't based at either endpoint. However, there are some cases where the same vertex is the site of more than one operation, and here it may be the case that order matters.  

We do know that $G_0$ is connected, because none of our three operations can disconnect the graph. And it has vertex degree $\leq 4$ because each operation reduces vertex degree.
\end{proof}

To construct all connected, embedded $4$-regular graphs from the embedded planar simple graphs with vertex degree $\leq 4$ requires us to invert this process.

 Construct the adjacency matrix of the graph $A$. At every vertex with degree $2$ or $1$, add a ``ghost'' loop edge with a weight of $0$,  At each edge 

plantri and the plantri theorem
reduction theorem for shadows
reduction for diagrams



\section{Classifying knot types}

homfly
mathematica
snappy

\section{Results}

giant pictures, compared with tait's classification
our distributions
monogon and bigon fractions
degree of alternatingness
universal properties? comparison with distribution from ERPs, lattice walks, and petaluma.

\section{Future Directions}

transitions, unknotting number and so forth.

\end{document}
