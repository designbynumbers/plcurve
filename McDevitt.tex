\documentclass{amsart}

\usepackage{amsrefs}
\setlength{\oddsidemargin}{0pt}
\setlength{\textwidth}{6.5in}

\begin{document}
In \cite {McDevittSlides}, Timothy McDevitt gives a formula for calculating
first derivative which minimizes the effects of noisy data on the value of
the derivative.  With some adaptation, that formula is:
$$y'_k = \frac{3}{n(n+1)(2n+1) \Delta t} \sum_{j=-n}^n j y_{k+j}$$
where $n$ in the formula above is $\frac{m-1}{2}$ in McDevitt's formula and
our $k$ is his $k - \frac{m+1}{2}$.

But what does it mean?  Well, let's look at the case $m=5 \Rightarrow n = 2$.
This says that
$$y'_k = \frac{1}{10 \Delta t} \left( -2 y_{k-2} - y_{k-1} + y_{k+1} + 2
  y_{k+2} \right).$$
Well, there are at least two ways to look at this.  First, let's see if we can
find derivatives in it.  Here's option one:
$$y'_k = \frac{1}{10} \left( 2 \frac{y_{k+2} - y_{k+1}}{\Delta t}
  + 3 \frac{y_{k+1} - y_k}{\Delta t} + 3 \frac{y_k - y_{k-1}}{\Delta t}
  + 2 \frac{y_{k-1} - y_{k-2}}{\Delta t} \right).$$
The other way to see it is
$$y'_k = \frac{1}{10} \left( 2^2 \frac{y_{k+2} - y_k}{2 \Delta t}
  + 1^2 \frac{y_{k+1} - y_k}{1 \Delta t}
  + 1^2 \frac{y_k - y_{k-1}}{1 \Delta t}
  + 2^2 \frac{y_k - y_{k-2}}{2 \Delta t} \right).$$

So there's a balance.  We weight the longer derivatives more because they
aren't so susceptible to random error.  However, the longer derivatives are
affected by second derivative effects (actually, third derivative effects,
since the left-and-right pairs will cancel out simple second derivative
effects).

So perhaps there is a way to pick $m$ ($n$) so that the total error is
minimized.  But that is a subject for another day.  Also, it seems like the
weighting is doable for unevenly-spaced points.  Suppose we have data for
$y(2.4)$, $y(4)$, $y(5.1)$, $y(5.3)$ and $y(6)$.  Then we guess that the
ideal derivative will be (taking $\Delta t = 1$):
\begin{align*}
y'(5.1) & =
  \frac{1}{(6 - 5.1)^2 + (5.3 - 5.1)^2 + (5.1 - 4)^2 + (5.1 - 2.4)^2} \\
& \hspace{0.3in}
  \left( (6 - 5.1)^2 \frac{y(6) - y(5.1)}{6 - 5.1} +
         (5.3 - 5.1)^2 \frac{y(5.3) - y(5.1)}{5.3 - 5.1} + \right. \\
& \hspace{0.5in}
  \left. (5.1 - 4)^2 \frac{y(5.1) - y(4)}{5.1 - 4} +
         (5.1 - 2.4)^2 \frac{y(5.1) - y(2.4)}{5.1 - 2.4} \right) \\
& = \frac{1}{9.35}
  \left( 0.81 \frac{y(6) - y(5.1)}{0.9} +
         0.04 \frac{y(5.3) - y(5.1)}{0.2} +
         1.21 \frac{y(5.1) - y(4)}{1.1} +
         7.29 \frac{y(5.1) - y(2.4)}{2.7} \right)
\end{align*}
Or, taking ($\Delta t = 2$):
\begin{align*}
y'(5.1) & =
  \frac{1}{(3 - 2.55)^2 + (2.65 - 2.55)^2 + (2.55 - 2)^2 + (2.55 - 1.2)^2} \\
& \hspace{0.3in}
  \left( (3 - 2.55)^2 \frac{y(6) - y(5.1)}{(3 - 2.55) 2} +
         (2.65 - 2.55)^2 \frac{y(5.3) - y(5.1)}{(2.65 - 2.55) 2} + \right. \\
& \hspace{0.5in}
  \left. (2.55 - 2)^2 \frac{y(5.1) - y(4)}{(2.55 - 2) 2} +
         (2.55 - 1.2)^2 \frac{y(5.1) - y(2.4)}{(2.55 - 1.2) 2} \right) \\
& = \frac{1}{2.3375}
  \left( 0.2025 \frac{y(6) - y(5.1)}{0.9} +
         0.01 \frac{y(5.3) - y(5.1)}{0.2} +
         0.3025 \frac{y(5.1) - y(4)}{1.1} +
         1.8225 \frac{y(5.1) - y(2.4)}{2.7} \right)
\end{align*}
Which appears to be the same thing.  That's encouraging.
\end{document}
