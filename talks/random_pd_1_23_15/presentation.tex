\documentclass[presentation]{beamer}
\usetheme{Antibes}
\usecolortheme{dolphin}
%\usepackage{amsmath}
\usepackage{amsthm}
\usepackage{amssymb}
\usepackage{graphicx}
\usepackage{fullpage}
\usepackage{enumerate}
\usepackage{array}
\usepackage{multicol}
%\usepackage{youngtab}
\usepackage{float}

\usepackage[utf8x]{inputenc}
\usepackage{default}
\usepackage{amssymb,amsfonts,amsbsy,amsmath}
%\usepackage{tkz-graph}
\usepackage{amsmath}
\usepackage{graphicx}

\newcommand{\mycirc}[1]
{\textcircled{\raisebox{-.8pt}{#1}}}

\newcommand{\mytwo}[2]
{\raisebox{1.1pt}{\scriptsize #1} \hspace{1pt} \raisebox{-1.1pt}{\scriptsize #2}}


%% Some useful operatorname declarations for nice texing.
\newcommand{\Aut}{\operatorname{Aut}}
\newcommand{\Inn}{\operatorname{Inn}}
\newcommand{\Ker}{\operatorname{Ker}}
\newcommand{\Stab}{\operatorname{Stab}}
\newcommand{\orb}[1]{\mathcal{O}_#1}
\newcommand{\lcm}{\operatorname{lcm}}
\newcommand{\U}{\operatorname U}
\newcommand{\cl}{\operatorname {cl}}

%% Stuff from King's .tex files
\newcommand{\untab}{\noindent \!\!\!\!\!\!\!\!\!}
\newcommand{\lra}{\longrightarrow}
\newcommand{\mb}[1]{\mathbf{#1}}
\newcommand{\gap}{\vspace{0.1in}}
\DeclareMathOperator{\grad}{grad}
\DeclareMathOperator{\curl}{curl}
\DeclareMathOperator{\GL}{GL}
\newcommand{\myline}
{\vspace{.2in}
  \begin{center}
    \rule{5in}{.7pt}
  \end{center}
  \vspace{.2in}}

\usepackage{parskip}

\setlength{\parindent}{2ex}
\setlength{\parskip}{2ex plus 1ex minus 1ex}
%% Good old "not sure if equal"
\newcommand{\qe}{\stackrel{?}{=}}

%% Some quickies for the big groups
\newcommand{\Q}{\mathbb Q}
\newcommand{\F}{\mathbb F}
\newcommand{\Qp}{\mathbb Q^+}
\newcommand{\Qn}{\mathbb Q^-}
\newcommand{\Qs}{\mathbb Q^*}
\newcommand{\R}{\mathbb R}
\newcommand{\Rp}{\mathbb R^+}
\newcommand{\Rn}{\mathbb R^-}
\newcommand{\Rs}{\mathbb R^*}
\newcommand{\Z}{\mathbb Z}
%\newcommand{\N}{\mathbb N}
%\newcommand{\C}{\mathbb C}

\newcommand{\Af}{\mathbb A}
\newcommand{\PP}{\mathbb P}

\newcommand{\GF}{\mathbb GF}

\newcommand{\gl}{\mathfrak g}
\newcommand{\bl}{\mathfrak b}
\newcommand{\tl}{\mathfrak t}
\newcommand{\pl}{\mathfrak p}
\newcommand{\nl}{\mathfrak n}

\DeclareMathOperator{\sh}{sh}

\newcommand{\Orb}{\mathcal O}
\newcommand{\Var}{\mathcal V}

\newcommand{\Irr}{\operatorname{Irr}}

\newcommand{\Brak}{[\cdot,\cdot]}

\newcommand{\NWN}{\nl \cap {^w\nl}}
\newcommand{\closeGNWN}{\overline{G \cdot (\NWN)}}

\newcommand{\N}{\mathbb N}
\newcommand{\C}{\mathbb C}

\newcommand{\GLn}{GL_n(\C)}
\newcommand{\gln}{\mathfrak {gl}_n}

\newcommand{\rank}{\operatorname{Rank}}

\newcommand{\bracket}{[\cdot, \cdot]}

\newcommand{\LRGfxSymArrowTwo}[2]{%
  \raisebox{1.3cm}{ \centering \parbox{.6cm}{ \centering%
    {\scriptsize $#1$}\\[-.23cm]$\longrightarrow$\\[-.3cm]%
    $\longleftarrow$\\[-.23cm] {\scriptsize $#2$}%
  } }%
}

% \newcommand{\LRGfxSymArrowTwo}[2]{
%   \raisebox{1.3cm}{
%     \begin{tabular}{c}
%       \centering%
%       {\scriptsize $#1$}\\[-.23cm]
%       $\longrightarrow$\\[-.3cm]%
%       $\longleftarrow$\\[-.23cm]
%       {\scriptsize $#2$}%
%     \end{tabular}
%   }
% }

\newcommand{\LRGfxSymArrow}[1]{
  \LRGfxSymArrowTwo{#1}{#1}
}
\usepackage{svg}
\usepackage{graphicx}
%\usepackage{mathrsfs}
%\usepackage{setspace}
%\usepackage{showkeys}
\usepackage{amsmath}
\usepackage{hyperref}
\usepackage{tikz}
\usepackage{pgfplots}
\usepackage{pgfplotstable}
\usepackage{array}
\pgfplotsset{compat=1.5.1}
\usetikzlibrary{trees,positioning,arrows,chains,knots,calc,decorations.markings}

\definecolor{beamerblue}{RGB}{234,233,243}
\definecolor{beamerviolet}{RGB}{47,23,132}
\definecolor{beamerliteviolet}{RGB}{137,127,207}

\tikzset{onslide/.code args={<#1>#2}{%
  \only<#1>{\pgfkeysalso{#2}} % \pgfkeysalso doesn't change the path
}}
\tikzset{temporal/.code args={<#1>#2#3#4}{%
  \temporal<#1>{\pgfkeysalso{#2}}{\pgfkeysalso{#3}}{\pgfkeysalso{#4}} % \pgfkeysalso doesn't change the path
}}

\tikzset{
  invisible/.style={opacity=0},
  visible on/.style={alt={#1{}{invisible}}},
  alt/.code args={<#1>#2#3}{%
    \alt<#1>{\pgfkeysalso{#2}}{\pgfkeysalso{#3}} % \pgfkeysalso doesn't change the path
  },
}

\newcommand{\Uhyp}{\mathcal U} \newcommand{\Vhyp}{\mathcal V}
\newcommand{\Rhyp}{\mathcal R}


\title{Random Planar Diagrams}
\institute[UGA]{University of Georgia}
\author[Chapman]{Harrison Chapman}
\date{Geometry seminar -- \today}

%\allowdisplaybreaks
%\usepackage[bookmarks,bookmarksopen,bookmarksdepth=4]{hyperref}

\newcommand{\so}[1]{\mathfrak {so}(#1)}

%\newtheorem{theorem}{Theorem}[section]
%\newtheorem{lemma}[theorem]{Lemma}
%\newtheorem{definition}[theorem]{Definition}
\newtheorem{proposition}[theorem]{Proposition}
%\newtheorem{corollary}[theorem]{Corollary}

%\renewcommand*\showkeyslabelformat[1]{\normalfont\tiny\ttfamily(#1)}

\let\oldemptyset\emptyset
\let\emptyset\varnothing

\DeclareMathOperator{\Arm}{Arm}
\DeclareMathOperator{\Pol}{Pol}
\DeclareMathOperator{\UP}{UP}
\DeclareMathOperator{\VP}{VP}
\DeclareMathOperator{\APol}{APol}
\DeclareMathOperator{\Diff}{Diff}
\DeclareMathOperator{\Sympl}{Sympl}
\DeclareMathOperator{\ev}{ev}
\DeclareMathOperator{\Ad}{Ad}
\DeclareMathOperator{\crit}{crit}
\DeclareMathOperator{\ind}{ind}
\DeclareMathOperator{\intr}{int}
\DeclareMathOperator{\Hom}{Hom}
\DeclareMathOperator{\Ext}{Ext}
\DeclareMathOperator{\codim}{codim}
\DeclareMathOperator{\Ann}{Ann}
\DeclareMathOperator{\im}{Im}
\DeclareMathOperator{\Int}{int}

\begin{document}
% \section{Symplectic Manifolds}
% \label{sec:sympmfd}
\newcommand{\Oh}[1]{\mathcal O (#1)}
\newcommand{\g}{\mathfrak g}
\newcommand{\ShSet}{\mathcal S}
\newcommand{\LnSet}{\mathcal L}
\newcommand{\sr}{/\!\!/}

\begin{frame}
\titlepage

\end{frame}

\section{Random Planar Diagrams}
\label{sec:randompd}
\newcommand{\Salpha}[1]{S^2_{\alpha_{#1}}}

\subsection{Definition}

\begin{frame}
  \frametitle{Planar diagrams and knots}
  A planar diagram is the usual way to express a space curve or knot
  in two dimensions. We will only
  consider \textit{oriented} planar diagrams, where components of
  links are given direction.
  \begin{figure}
    \centering
    \begin{tikzpicture}
      \begin{scope}[very thick,decoration={
          markings,
          mark=between positions .2 and 1 step 0.2 with {\arrow{>}}}
        ]
        \def\foil{5}
        \foreach \brk in {1,...,\foil} {
          \begin{scope}[rotate=\brk * 360/\foil]
            \node[transform shape, knot crossing, inner sep=1.5pt] (k\brk) at (0,-1) {};
          \end{scope}
        }
        \draw[ultra thick,black,postaction={decorate}] (0,0) \foreach \brk in
        {1,...,\foil} {let \n0=\brk, \n1={int(Mod(\brk,\foil)+1)},
          \n2={int(Mod(\brk+1,\foil)+1)} in (k\n0) .. controls
          (k\n0.16 south east) and (k\n1.16 south west)
          .. (k\n1.center) .. controls (k\n1.4 north east) and (k\n2.4
          north west) .. (k\n2)};
      \end{scope}
    \end{tikzpicture}
    \caption{A typical planar diagram for the knot $5_1$.}
  \end{figure}
\end{frame}

\begin{frame}
  \frametitle{Planar diagrams as graphs}
  \begin{definition}
    A \textbf{planar diagram} is an isotopy class of 4-valent embedded
    planar directed multigraphs
    \begin{itemize}
    \item together with information at each vertex as to which opposing
      pair of edges constitutes the over strand and under strand
      (\textbf{crossing sign}),
    \item and so that at every vertex, opposite
      edges do not both point towards or away from the vertex.
    \end{itemize}

  \end{definition}


  Two planar diagrams are equivalent if they differ only by
  \textbf{diagram isotopy}, wherein both the underlying planar digraph
  embeddings are isomorphic, and crossing signs agree.
\end{frame}

\begin{frame}
  \frametitle{A planar diagram}
  \begin{figure}
    \centering
    \begin{minipage}[c]{.4\linewidth}
      \vspace{0pt}
      \begin{tikzpicture}
        \begin{scope}[very thick,decoration={
            markings,
            mark=between positions 1/3 and 1 step 1/3 with {\arrow{>}}}
          ]
          \def\foil{3}
          \foreach \brk in {1,...,\foil} {
            \begin{scope}[rotate=\brk * 360/\foil]
              \node[transform shape, knot crossing, inner sep=1.5pt] (k\brk) at (0,-1) {};
            \end{scope}
          }
          \draw[ultra thick,black,postaction={decorate}] (0,0) \foreach \brk in
          {1,...,\foil} {let \n0=\brk, \n1={int(Mod(\brk,\foil)+1)},
            \n2={int(Mod(\brk+1,\foil)+1)} in (k\n0) .. controls
            (k\n0.16 south east) and (k\n1.16 south west)
            .. (k\n1.center) .. controls (k\n1.4 north east) and (k\n2.4
            north west) .. (k\n2)};
        \end{scope}
      \end{tikzpicture}
    \end{minipage}
    \hfill
    \begin{minipage}[c]{.4\linewidth}
      \vspace{0pt}
      \begin{tikzpicture}
        \begin{scope}[very thick,decoration={
            markings,
            mark=between positions 1/3 and 1 step 1/3 with {\arrow{>}}}
          ]
          \def\foil{3}
          \foreach \brk in {1,...,\foil} {
            \begin{scope}[rotate=\brk * 360/\foil]
              \node[transform shape, knot crossing, inner sep=1.5pt] (k\brk) at (0,-1) {\textbullet};
            \end{scope}
            \node[blue] at (k\brk.south) {\large $-$};
          }
          \draw[ultra thick,black,->] (k1) --(k2);
          \draw[ultra thick,black,->] (k2) --(k3);
          \draw[ultra thick,black,->] (k3) --(k1);
          \draw[ultra thick,black,->] (k1) to[bend right=90] (k2);
          \draw[ultra thick,black,->] (k2) to[bend right=90] (k3);
          \draw[ultra thick,black,->] (k3) to[bend right=90] (k1);
        \end{scope}
      \end{tikzpicture}
    \end{minipage}
    \caption{Planar diagram for a typical trefoil knot in a standard
      presentation, and as an annotated directed multigraph.}
  \end{figure}

\end{frame}

\begin{frame}
  \frametitle{Crossing sign}
  Crossing signs are determined by the right hand rule.
  \begin{figure}
    \centering
    \begin{minipage}[c]{.4\linewidth}
      \vspace{0pt}
      \centering
      \begin{tikzpicture}
        \draw[knot=black,ultra thick,decoration={markings, mark=at
          position 0.05 with {\arrow[black,ultra thick]{<}}},postaction={decorate}] (-1,1) -- (1,-1);
        \draw[knot=black,ultra thick,decoration={markings, mark=at
          position 1 with {\arrow[black,ultra
            thick]{>}}},postaction={decorate}] (-1,-1) -- (1,1);
        \node at (0,-.5) {\large $+$};
      \end{tikzpicture}
    \end{minipage}
    \hfill
    \begin{minipage}[c]{.4\linewidth}
      \vspace{0pt}
      \centering
      \begin{tikzpicture}
        \draw[knot=black,ultra thick,decoration={markings, mark=at
          position 1 with {\arrow[black,ultra thick]{>}}},postaction={decorate}] (-1,-1) -- (1,1);
        \draw[knot=black,ultra thick,decoration={markings, mark=at
          position 0.05 with {\arrow[black,ultra
            thick]{<}}},postaction={decorate}] (-1,1) -- (1,-1);
        \node at (0,-.5) {\large $-$};
      \end{tikzpicture}
    \end{minipage}
    \caption{Positive and negative crossings.}
  \end{figure}
\end{frame}

\begin{frame}
  \frametitle{Diagram isotopy}
  \begin{figure}
    \centering
    \begin{minipage}[c]{.4\linewidth}
      \vspace{0pt}
       \begin{tikzpicture}
        \begin{knot}[
          consider self intersections,
          scale=1.5,
          flip crossing/.list={2,3},
          clip width=5]
          \strand[black,decoration={markings, mark=between
          positions 0 and 1 step 1/6 with {\arrow[black,
            thick]{>}}},postaction={decorate}]
          (90:1) to[out=180,in=-120,looseness=2]
          (-30:1) to[out=60,in=120,looseness=2]
          (210:1) to[out=-60,in=0,looseness=2] (90:1);
        \end{knot}
      \end{tikzpicture}%\vspace{-26pt}
    \end{minipage}
    \hfill
    \begin{minipage}[c]{.4\linewidth}
      \vspace{0pt}
       \begin{tikzpicture}
        \begin{knot}[
          consider self intersections,
          scale=1.5,
          clip width=5]
          \strand[black,decoration={markings, mark=between
          positions 0 and 1 step 1/6 with {\arrow[black,
            thick]{<}}},postaction={decorate}]
          (90:1) to[out=180,in=-120,looseness=2]
          (-30:1) to[out=60,in=120,looseness=2]
          (210:1) to[out=-60,in=0,looseness=2] (90:1);
        \end{knot}
      \end{tikzpicture}
    \end{minipage}
    \caption{The two diagrams are related by flipping the inside with
      the out and a rotation.}
  \end{figure}

\end{frame}

\begin{frame}
  \frametitle{Planar diagram shadows}
  A \textbf{diagram shadow} is an isomorphism class of 4-valent
  embedded planar multigraphs, up to planar graph isomorphim.

  \begin{figure}
    \centering\
    \begin{minipage}[c]{.4\linewidth}
      \vspace{0pt}
      \centering
      \begin{tikzpicture}
        \begin{knot}[scale=1.2]
          \strand[ultra thick,black]
          (90:1) to[out=180,in=-120,looseness=2]
          (-30:1) to[out=60,in=120,looseness=2]
          (210:1) to[out=-60,in=0,looseness=2] (90:1);
        \end{knot}
      \end{tikzpicture}\vspace{-26pt}
    \end{minipage}
    \hfill
    \begin{minipage}[c]{.4\linewidth}
      \vspace{0pt}
      \centering
      \begin{tikzpicture}
        \begin{scope}[very thick,decoration={
            markings,
            mark=between positions 1/3 and 1 step 1/3 with {\arrow{>}}}
          ]
          \def\foil{3}
          \foreach \brk in {1,...,\foil} {
            \begin{scope}[rotate=\brk * 360/\foil]
              \node[transform shape, knot crossing, inner sep=1.5pt] (k\brk) at (0,-1) {\textbullet};
            \end{scope}
          }
          \draw[ultra thick,black] (k1) --(k2);
          \draw[ultra thick,black] (k2) --(k3);
          \draw[ultra thick,black] (k3) --(k1);
          \draw[ultra thick,black] (k1) to[bend right=90] (k2);
          \draw[ultra thick,black] (k2) to[bend right=90] (k3);
          \draw[ultra thick,black] (k3) to[bend right=90] (k1);
        \end{scope}
      \end{tikzpicture}
    \end{minipage}
    \caption{A diagram shadow viewed both as a drawing and as a graph.}
    \label{fig:diashadow}
  \end{figure}
\end{frame}

\begin{frame}
  \frametitle{Counting planar diagrams}
  \begin{block}{Finiteness}
   The spaces of planar
   diagrams or diagram shadows with [at most] $n$ crossings is
    \textit{finite}.
  \end{block}

  \begin{table}
    \centering
    \begin{tabular}{r|cc}
      n & \# knot shadows & \# knot diagrams \\
      \hline
      3 & 6 & $<96$ \\
      4 & 16 & $<512$ \\
      5 & 63 & $<4032$ \\
      6 & 302 & $<38656$ \\
      7 & 1756 & $<449536$ \\
      8 & 11621 & $<5949952$ \\
      9 & $<193903$ & $<193903 * 2^{10}$
    \end{tabular}
    \caption{Counts and bounds on knot diagrams and shadows. Numbers
      are large, but finite.}
    \label{tab:counts}
  \end{table}
\end{frame}

\begin{frame}
  \frametitle{Exploring the space of diagrams}
  \begin{figure}
    \centering
    \includegraphics[width=\textwidth,height=\textheight,keepaspectratio]{epsgroup_0_collage.eps}
    \caption{Planar diagrams}
    \label{fig:planarcollage0}
  \end{figure}
\end{frame}

\begin{frame}
  \frametitle{Exploring the space of diagrams}
  \begin{figure}
    \centering
    \includegraphics[width=\textwidth,height=\textheight,keepaspectratio]{epsgroup_1_collage.eps}
    \caption{Planar diagrams}
    \label{fig:planarcollage1}
  \end{figure}
\end{frame}

\begin{frame}
  \frametitle{Exploring the space of diagrams}
  \begin{figure}
    \centering
    \includegraphics[width=\textwidth,height=\textheight,keepaspectratio]{epsgroup_10_collage.eps}
    \caption{Planar diagrams}
    \label{fig:planarcollage10}
  \end{figure}
\end{frame}

\begin{frame}
  \frametitle{Exploring the space of diagrams}
  \begin{figure}
    \centering
    \includegraphics[width=\textwidth,height=\textheight,keepaspectratio]{epsgroup_16_collage.eps}
    \caption{Planar diagrams. A map of all shadows with between 3 and
      6 crossings \href{http://prezi.com/s5te-8obcfgq/?utm_campaign=share&utm_medium=copy&rc=ex0share}{is here.}}
    \label{fig:planarcollage16}
  \end{figure}
\end{frame}


\subsection{A Complete List}

\begin{frame}
  \frametitle{Enumeration of appropriate planar graphs}
  The first step is to generate a complete list of diagram shadows.

  \begin{block}{plantri}
    A program \texttt{plantri} by McKay and Brinkmann is able to
    generate all planar embedded graphs (not multigraphs) with
    arbitrary numbers of vertices.
  \end{block}
  We then omit all such graphs with vertices of degree $> 4$.
\end{frame}

\begin{frame}
  \frametitle{Shadows from planar graphs}
  Given an appropriate graph $G$ from \texttt{plantri} and a vertex $V
  \in G$, there are two
  different operations which are repeated until the vertex (and then
  all vertices) has degree 4:

  \begin{enumerate}
  \item Adding a self-loop to $V$ inside one of the faces of the
    embedding. This increases the degree of the vertex by 2.
  \item Doubling an edge adjacent to $V$ in $G$. This increases the
    degrees of the two connected vertices each by 1.
  \end{enumerate}

  This process produces a number of (possibly not all unique) diagram
  shadows.
\end{frame}

\begin{frame}
  \frametitle{Self loop addition}
  \begin{figure}
    \centering
    \begin{minipage}[c]{.32\linewidth}
      \vspace{0pt}
      \begin{tikzpicture}
        \begin{scope}[very thick,decoration={
            markings,
            mark=between positions 1/3 and 1 step 1/3 with {\arrow{>}}}
          ]
          \def\foil{3}
          \foreach \brk in {1,...,\foil} {
            \begin{scope}[rotate=\brk * 360/\foil]
              \node[transform shape, knot crossing, inner sep=1.5pt] (k\brk) at (0,-1) {\textbullet};
            \end{scope}
          }
            \node[red] at (k3) {\large\textbullet};
          \draw[ultra thick,black] (k1) to[bend right=90] (k2);
          \draw[ultra thick,black] (k2) to[bend right=90] (k3);
          \draw[ultra thick,black] (k3) to[bend right=90] (k1);
        \end{scope}
      \end{tikzpicture}
    \end{minipage}
    \hfill
    \begin{minipage}[c]{.32\linewidth}
      \vspace{0pt}
      \begin{tikzpicture}
        \begin{scope}[very thick,decoration={
            markings,
            mark=between positions 1/3 and 1 step 1/3 with {\arrow{>}}}
          ]
          \def\foil{3}
          \foreach \brk in {1,...,\foil} {
            \begin{scope}[rotate=\brk * 360/\foil]
              \node[transform shape, knot crossing, inner sep=1.5pt] (k\brk) at (0,-1) {\textbullet};
            \end{scope}
          }
            \node[red] at (k3) {\large\textbullet};
          \draw[ultra thick,black] (k1) to[bend right=90] (k2);
          \draw[ultra thick,black] (k2) to[bend right=90] (k3);
          \draw[ultra thick,black] (k3) to[bend right=90] (k1);
          \draw[ultra thick,blue] (k3) to[out=135,in=45,looseness=10] (k3);
        \end{scope}
      \end{tikzpicture}
    \end{minipage}
    \hfill
        \begin{minipage}[c]{.32\linewidth}
      \vspace{0pt}
      \begin{tikzpicture}
        \begin{scope}[very thick,decoration={
            markings,
            mark=between positions 1/3 and 1 step 1/3 with {\arrow{>}}}
          ]
          \def\foil{3}
          \foreach \brk in {1,...,\foil} {
            \begin{scope}[rotate=\brk * 360/\foil]
              \node[transform shape, knot crossing, inner sep=1.5pt] (k\brk) at (0,-1) {\textbullet};
            \end{scope}
          }
            \node[red] at (k3) {\large\textbullet};
          \draw[ultra thick,black] (k1) to[bend right=90] (k2);
          \draw[ultra thick,black] (k2) to[bend right=90] (k3);
          \draw[ultra thick,black] (k3) to[bend right=90] (k1);
          \draw[ultra thick,blue] (k3) to[out=-135,in=-45,looseness=10] (k3);
        \end{scope}
      \end{tikzpicture}
    \end{minipage}
    \caption{Addition of a self loop. From the graph on the left, with
      red selected vertex, the rightmost two graphs are produced.}
    \label{fig:selfloop}
  \end{figure}
\end{frame}

\begin{frame}
  \frametitle{Edge doubling}
  \begin{figure}
    \centering
    \begin{minipage}[c]{.32\linewidth}
      \vspace{0pt}
      \begin{tikzpicture}
        \begin{scope}[very thick,decoration={
            markings,
            mark=between positions 1/3 and 1 step 1/3 with {\arrow{>}}}
          ]
          \def\foil{3}
          \foreach \brk in {1,...,\foil} {
            \begin{scope}[rotate=\brk * 360/\foil]
              \node[transform shape, knot crossing, inner sep=1.5pt] (k\brk) at (0,-1) {\textbullet};
            \end{scope}
          }
            \node[red] at (k3) {\large\textbullet};
          \draw[ultra thick,black] (k1) to[bend right=90] (k2);
          \draw[ultra thick,black] (k2) to[bend right=90] (k3);
          \draw[ultra thick,black] (k3) to[bend right=90] (k1);
        \end{scope}
      \end{tikzpicture}
    \end{minipage}
    \hfill
    \begin{minipage}[c]{.32\linewidth}
      \vspace{0pt}
      \begin{tikzpicture}
        \begin{scope}[very thick,decoration={
            markings,
            mark=between positions 1/3 and 1 step 1/3 with {\arrow{>}}}
          ]
          \def\foil{3}
          \foreach \brk in {1,...,\foil} {
            \begin{scope}[rotate=\brk * 360/\foil]
              \node[transform shape, knot crossing, inner sep=1.5pt] (k\brk) at (0,-1) {\textbullet};
            \end{scope}
          }
            \node[red] at (k3) {\large\textbullet};
          \draw[ultra thick,black] (k1) to[bend right=90] (k2);
          \draw[ultra thick,black] (k2) to[bend right=90] (k3);
          \draw[ultra thick,black] (k3) to[bend right=90] (k1);
          \draw[ultra thick,blue] (k3) to[bend left=45] (k1);
        \end{scope}
      \end{tikzpicture}
    \end{minipage}
    \hfill
        \begin{minipage}[c]{.32\linewidth}
      \vspace{0pt}
      \begin{tikzpicture}
        \begin{scope}[very thick,decoration={
            markings,
            mark=between positions 1/3 and 1 step 1/3 with {\arrow{>}}}
          ]
          \def\foil{3}
          \foreach \brk in {1,...,\foil} {
            \begin{scope}[rotate=\brk * 360/\foil]
              \node[transform shape, knot crossing, inner sep=1.5pt] (k\brk) at (0,-1) {\textbullet};
            \end{scope}
          }
            \node[red] at (k3) {\large\textbullet};
          \draw[ultra thick,black] (k1) to[bend right=90] (k2);
          \draw[ultra thick,black] (k2) to[bend right=90] (k3);
          \draw[ultra thick,black] (k3) to[bend right=90] (k1);
          \draw[ultra thick,blue] (k2) to[bend left=45] (k3);
        \end{scope}
      \end{tikzpicture}
    \end{minipage}
    \caption{Doubling of an edge. From the graph on the left, with
      red selected vertex, the rightmost two graphs are produced.}
    \label{fig:edgedouble}
  \end{figure}

\end{frame}

\begin{frame}
  \frametitle{Ordering shadows}
  The ultimate list of diagram shadows is ordered by how the algorithm
  produces them:
  \begin{itemize}
  \item The order in which \texttt{plantri} outputs its planar graphs
    is fixed.
  \item The algorithm which creates planar diagram shadows from the
    low-valence planar graph embeddings has order.
  \end{itemize}
\end{frame}

\begin{frame}
  \frametitle{Exploding shadows}
  Given a diagram shadow, two sets of binary choices must be made to
  determine a planar diagram:
  \begin{enumerate}
  \item Edges must be given a direction.
  \item Crossings must be given a sign.
  \end{enumerate}
\end{frame}

\begin{frame}
  \frametitle{Orienting components}
  \begin{block}{One edge per component}
    The constraint on edge orientations in the definition of planar
    diagrams means that orienting all edges is equivalent to orienting
    one edge per component.
  \end{block}

  For knots, there is only one component to orient.
\end{frame}

\begin{frame}
  \frametitle{Assigning crossing signs}
  For any given crossing, there are two valid signs; $+$ and $-$.

  \begin{block}{Bound on diagrams for a shadow}
    For a $k$-crossing knot (1-component) diagram, there are 2 choices
    of orientation. There are then $2^k$ total choices for crossing
    signs. So there are at most $2^{k+1}$ planar diagram isotopy
    classes for each shadow.
  \end{block}
\end{frame}

\subsection{Classification of knot type}

\begin{frame}
  \frametitle{The HOMFLY polynomial}
  \begin{block}{HOMFLY polynomial}
    The \textbf{HOMFLY polynomial} is a knot-invariant polynomial of
    an oriented planar diagram defined by the following skein
    relation:

    \[ zP\left(
    \begin{tikzpicture}[baseline={(0,3pt)}]
      \draw[thick,->] (0,0) to[bend right=45] (0,12pt);
      \draw[thick,->] (12pt,0) to[bend left=45] (12pt,12pt);
    \end{tikzpicture}\right)
  = aP\left(\begin{tikzpicture}[baseline={(0,3pt)}]
        \draw[knot=black,background color=beamerblue,thick,decoration={markings, mark=at
          position 0.13 with {\arrow[black,thick]{<}}},postaction={decorate}] (0,12pt) -- (12pt,0);
        \draw[knot=black,background color=beamerblue,thick,decoration={markings, mark=at
          position 1 with {\arrow[black,
            thick]{>}}},postaction={decorate}] (0,0) -- (12pt,12pt);
    \end{tikzpicture}\right) + a^{-1}P\left(\begin{tikzpicture}[baseline={(0,3pt)}]
      \draw[knot=black,background color=beamerblue,thick,decoration={markings, mark=at
          position 1 with {\arrow[black,
            thick]{>}}},postaction={decorate}] (0,0) -- (12pt,12pt);
      \draw[knot=black,background color=beamerblue,thick,decoration={markings, mark=at
          position 0.13 with {\arrow[black,thick]{<}}},postaction={decorate}] (0,12pt) -- (12pt,0);
    \end{tikzpicture}\right),
\qquad P(0_1) = 1.\]
  \end{block}

  For a knot $K$,
  \[ P(K)(a,z) = P(K^*)(a^{-1},z)\]
  For two knots $K, L$,
  \[ P(K \# L) = P(K)P(L) \text{ and } P(K \cup L) = \left(\frac{a+a^{-1}}{z}\right)P(K)P(L)\]
\end{frame}

\begin{frame}
  \frametitle{Classifying knots with small crossing number}
  All knots (both prime and composite) with a crossing number of 7 or
  smaller are classified entirely by their HOMFLY polynomial.

  \begin{table}
    \centering
    \begin{tabular}{r|l}
      Knot type & HOMFLY polynomial \\
      \hline
      $0_1$ & $1$ \\
      $3_1$ & $-2a^{2} + a^{2}z^{2} -a^{4}$ \\
      $3_1^*$ & $-a^{-4} -2a^{-2} + a^{-2}z^{2}$ \\
      $4_1$ & $-a^{-2} -1 + z^{2} -a^{2}$ \\
      $5_1$ & $3a^{4} -4a^{4}z^{2} + a^{4}z^{4} + 2a^{6} -a^{6}z^{2}$ \\
      $5_2$ & $-a^{2} + a^{2}z^{2} + a^{4} -a^{4}z^{2} + a^{6}$ \\
      $3_1 \# 3_1$ & $4a^{4} -4a^{4}z^{2} + a^{4}z^{4} + 4a^{6} -2a^{6}z^{2} + a^{8}$ \\
      $3_1 \# 3_1^*$ & $2a^{-2} -a^{-2}z^{2} + 5 -4z^{2} + z^{4} + 2a^{2} -a^{2}z^{2}$ \\
      $6_1$ & $-a^{-2} + z^{2} + a^{2} -a^{2}z^{2} + a^{4}$ \\
    \end{tabular}
    \caption{Some knot types and their HOMFLY polynomial}
    \label{tab:homflys}
  \end{table}
\end{frame}

\begin{frame}
  \frametitle{HOMFLY polynomial collisions}
  \begin{block}{Negative amphichiral $8_{17}$}
    In 8 crossings; the
    knots $8_{17}$ and $8_{17}^*$ have the same polynomial
    {\footnotesize\begin{align*}
      & -a^{-2} + 2a^{-2}z^{2} -a^{-2}z^{4} -1 + 5z^{2} -4z^{4} +
        z^{6} -a^{2} + 2a^{2}z^{2} -a^{2}z^{4}
    \end{align*}}
    but are \textbf{different} as oriented knots.
  \end{block}

  \begin{block}{Prime/composite collision}
    In 9-crossing knots; $9_{12}$ and $4_1 \# 5_2^*$ both
    have HOMFLY polynomial
    {\footnotesize\begin{align*}
      &-a^{-8} -2a^{-6} + 2a^{-6}z^{2} -a^{-4} + a^{-4}z^{2} -a^{-4}z^{4} -a^{-2}z^{2} + a^{-2}z^{4} + 1 -z^{2}.
    \end{align*}}
  \end{block}

\end{frame}

\begin{frame}
  \frametitle{The peculiar $8_{17}$}
  \begin{figure}
    \centering
    \includegraphics[height=.7\textheight,keepaspectratio]{8_17_3d.jpg}
    \caption{$8_{17}$ is not equal to its inverse if it is oriented.}
    \label{fig:8_17}
  \end{figure}
\end{frame}

\begin{frame}
  \frametitle{Distinguishing $9_{12}$ and $4_1 \# 5_2^*$}
  The (prime) satellite knot of minimal crossing number has crossing
  number 13; hence, no complements of prime knots with crossing number
  $<13$ have any incompressible non boundary-parallel tori.

  \begin{block}{Help from SnapPy}
    The software package \texttt{SnapPy} has a method
    \texttt{splitting\_surfaces} which finds such surfaces; it finds
    none for $9_{12}$, but two for $4_1 \# 5_2^*$.
  \end{block}
  The knots $9_{12}$ and $4_1 \# 5_2^*$ are also distinguished by
  their Kauffman polynomial (not bracket).
\end{frame}

\section{Analysis of Data}
\label{sec:data}

\subsection{Knotting probabilities}

\begin{frame}
  \frametitle{Experimental ratios}
  \begin{table}
    \centering\scriptsize
    \pgfkeys{/pgf/number format/.cd,fixed,fixed zerofill,precision=5}
    \pgfplotstabletypeset[
    columns={n,0.1,3.1,4.1,5.1,5.2,6.1,6.2},
    columns/n/.style={precision=0, column name=$n$, column type=r|},
    columns/0.1/.style={column name=$0_1$},
    columns/3.1/.style={column name=$3_1$},
    columns/4.1/.style={column name=$4_1$},
    columns/5.1/.style={column name=$5_1$},
    columns/5.2/.style={column name=$5_2$},
    columns/6.1/.style={column name=$6_1$},
    columns/6.2/.style={column name=$6_2$},
    every first row/.style={before row={\hline}},
    ]{knot_freq.tsv}
    \caption{Ratios of knots appearing in planar diagrams with $\le n$
    crossings.}
  \end{table}
\end{frame}

\begin{frame}
  \frametitle{Experimental ratios (unknots in diagrams)}
  \begin{figure}
    \centering
    \begin{tikzpicture}[scale=.75]
      \begin{axis}[
        ymode=log,
        title={Ratio of unknots in $\le n$-crossing diagrams (log scale)},
        xlabel={Max \# crossings in diagram},
        ylabel={Ratio of unknots},
        ]
        \addplot table[x=n,y=0.1] {knot_freq.tsv};
      \end{axis}
    \end{tikzpicture}

    \caption{Unknot ratio decreases exponentially.}
    \label{fig:unkdecay}
  \end{figure}
\end{frame}

\begin{frame}
  \frametitle{Experimental ratios (unknots in diagrams)}
  \begin{figure}
    \centering
    \begin{tikzpicture}[scale=.75]
      \begin{axis}[
        ymode=log,
        title={Ratio of unknots in $\le n$-crossing diagrams (log scale)},
        xlabel={Max \# crossings in diagram},
        ylabel={Ratio of unknots},
        legend pos=south west,
        ]
        \addplot table[x=n,y=0.1] {knot_freq.tsv};
        \addlegendentry{$0_1$}
        \addplot table[x=n,y={create col/linear regression={y=0.1}}]
        {knot_freq.tsv};
        \addlegendentry{slope
          $\pgfmathprintnumber{\pgfplotstableregressiona}$}
      \end{axis}
    \end{tikzpicture}

    \caption{Unknot ratio decreases exponentially.}
    \label{fig:unkregress}
  \end{figure}
\end{frame}

\begin{frame}
  \frametitle{Experimental ratios (knotting)}
  \begin{figure}
    \centering
    \begin{tikzpicture}[scale=.75]
      \begin{axis}[
        ymode=log,
        title={Ratios of knots in $\le n$-crossing diagrams (log scale)},
        xlabel={Max \# crossings in diagram},
        ylabel={Ratios of knots},
        legend pos=south west,
        legend entries={$3_1$, $4_1$, $5_1$, $5_2$},
        ]
        \addplot table[x=n,y=3.1] {knot_freq.tsv};
        \addplot table[x=n,y=4.1] {knot_freq.tsv};
        \addplot table[x=n,y=5.1] {knot_freq.tsv};
        \addplot table[x=n,y=5.2] {knot_freq.tsv};
      \end{axis}
    \end{tikzpicture}

    \caption{All ratios of knot types are still increasing.}
    \label{fig:kgrow1}
  \end{figure}
\end{frame}

\begin{frame}
  \frametitle{Experimental ratios (knotting)}
  \begin{figure}
    \centering
      \tikzset{mark=*}
    \begin{tikzpicture}[scale=.75]
      \begin{axis}[
        ymode=log,
        title={Ratios of knots in $\le n$-crossing diagrams (log scale)},
        xlabel={Max \# crossings in diagram},
        ylabel={Ratios of knots},
        cycle list name=color list,
        legend style={font=\tiny},
        legend pos=south west,
        legend entries={$3_1$, $4_1$, $5_1$, $5_2$, $6_1$, $6_2$, $3_1
          \# 3_1$, $3_1 \# 3_1^*$, $7_1$, $7_2$, $7_3$, $7_4$, $7_5$,
          $7_6$, $7_7$},
        ]
        \addplot table[x=n,y=3.1] {knot_freq.tsv};
        \addplot table[x=n,y=4.1] {knot_freq.tsv};
        \addplot table[x=n,y=5.1] {knot_freq.tsv};
        \addplot table[x=n,y=5.2] {knot_freq.tsv};
        \addplot table[x=n,y=6.1] {knot_freq.tsv};
        \addplot table[x=n,y=6.2] {knot_freq.tsv};
        \addplot table[x=n,y=3.1c3.1] {knot_freq.tsv};
        \addplot table[x=n,y=3.1c3.1x] {knot_freq.tsv};
        \addplot table[x=n,y=7.1] {knot_freq.tsv};
        \addplot table[x=n,y=7.2] {knot_freq.tsv};
        \addplot table[x=n,y=7.3] {knot_freq.tsv};
        \addplot table[x=n,y=7.4] {knot_freq.tsv};
        \addplot table[x=n,y=7.5] {knot_freq.tsv};
        \addplot table[x=n,y=7.6] {knot_freq.tsv};
        \addplot table[x=n,y=7.7] {knot_freq.tsv};
      \end{axis}
    \end{tikzpicture}

    \caption{Ratios of knot types are still increasing.}
    \label{fig:kgrow2}
  \end{figure}
\end{frame}

\begin{frame}
  \frametitle{Knot type and random polygons}
  \begin{figure}
    \centering
    \includegraphics[height=0.6\textheight,keepaspectratio]{grp_kprobs.png}
    \caption{Random knotting probability for the knots $3_1$, $3_1 \#
      3_1$, and $3_1 \# 3_1 \# 3_1$ in Gaussian random polygons with
      $N$ edges [Deguchi-Tsurusaki 1998]}
    \label{fig:degtsugraph}
  \end{figure}
\end{frame}

\begin{frame}
  \frametitle{Knotting in random planar diagrams}
  \begin{figure}
    \centering
    \begin{tikzpicture}[scale=.7]
      \begin{axis}[title={Ratios of knots},
        xlabel={Max \# crossings in diagram},
        ylabel={Ratios of knots},
        legend pos=north west,
        legend entries={$3_1$, $3_1 \# 3_1$},
        ]
        \addplot table[x=n,y=3.1] {knot_freq.tsv};
        \addplot table[x=n,y=3.1c3.1] {knot_freq.tsv};
      \end{axis}
    \end{tikzpicture}

    \caption{At $n=8$, the ratio of $3_1$ is 0.071; in Gaussian random
    polygons, the trefoil appears with probability approximately 0.07
    when $N \approx 50$.}
    \label{fig:kgrow1}
  \end{figure}
\end{frame}

\begin{frame}
  \frametitle{Unknotting in random confined polygons}
  \begin{figure}
    \centering
    \includegraphics[width=.9\textwidth,keepaspectratio]{conf_kprobs.png}
    \caption{As a random polygon becomes more confined (i.e. the
      confinement radius $R$ shrinks), unknotting probability
      decreases. Right: the blue data has $R = 1$, and the maroon has
      $R = 1.5$. [Diao-Ernst-Ziegler 2014] }
    \label{fig:diaoconf}
  \end{figure}
\end{frame}

\subsection{Knot distance}

\begin{frame}
  \frametitle{Distance between knots}
  A notion of distance between knot types is the following.

  \begin{block}{Knot distance}
    The distance between two knots $K$ and $L$ is the minimum number
    of crossing sign toggles required to produce $L$ from $K$.
  \end{block}

  The \textbf{unknotting number} of a knot $K$ is the distance from
  $K$ to the unknot $0_1$.
\end{frame}

\begin{frame}
  \frametitle{Topoisomerase}
  An immediate application of knot distance is found in the enzyme
  \textit{topoisomerase}, which aids in the replication of DNA by cutting a
  strand of a DNA double helix, pulling it through the other, and gluing it back.
  \begin{figure}\centering
    \includegraphics[width=\textwidth,keepaspectratio]{topoisomerase.jpg}
    \caption{Topoisomerase brings strands of DNA closer to unlinking.}
    \label{fig:topoisomerase}
  \end{figure}

\end{frame}

\begin{frame}
  \frametitle{Known knot distances}
  Isabel Darcy has compiled tables of known ranges for the distances between
  knot types. Even in the first few rows there are unknowns.

    \begin{table}
      \centering\small
      \begin{tabular}{r|ccccccccccc}
        &$0_1$ & $3_1$ & $4_1$ & $5_1$ & $5_2$ & $6_1$ & $6_2$ & $6_3$ & $3_1 \# 3_1$ & $3_1 \# 3_1^*$ \\%& $7_1$ & \\
        \hline
        $3_1$ & 1 & 0 & 2 & 1 & 1 & 2 & 1 & 1 & 1 & 1 \\%& 2\\ % & 2 & 3 & 2-3 \\
        $3_1^*$ & 1 & 2 & 2 & 3 & 2 & 2 & 2 & 1 & 3 & 1 \\%& 4 \\% 2 & 2 & 1 \\
        $4_1$ & 1 & 2 & 0 & 2-3 & 2 & 1 & 1 & 2 & 2-3 & 2-3 \\%& 3-4 \\% 2 & 2-3 & 2-3 \\
        $5_1$ & 2 & 1 & 2-3 & 0 & 1 & 2-3 & 2 & 2 & 2 & 2 \\%& 1 \\% 2 & 4 & 3-4 \\
        $5_1^*$ & 2 & 3 & 2-3 & 4 & 3 & 2-3 & 3 & 2 & 4 & 2 \\%& 5 \\%& 3 & 1 & 2 \\
        $5_2$ & 1 & 1 & 2 & 1 & 0 & 2 & 2 & 2 & 2 & 2 \\%& 2 \\%& 1 & 3 & 2-3 \\
        $5_2^*$ & 1 & 2 & 2 & 3 & 2 & 2 & 2 & 2 & 3 & 2 \\%& 4 \\%& 2 & 1 & 1 \\
        $6_1$ & 1 & 2 & 1 & 2-3 & 2 & 0 & 1 & 2 & 2-3 & 1-3 \\%& 3-4 \\%& 2 & 2-3 & 2-3 \\
        $6_1^*$ & 1 & 2 & 1 & 2-3 & 2 & 1 & 2 & 2 & 2-3 & 1-3 \\%& 3-4 \\%& 2 & 2-3 & 2-3 \\
      \end{tabular}
      \caption{A piece of the knot distance tabulation by Darcy.}
      \label{fig:disttable}
    \end{table}
\end{frame}

\begin{frame}
  \frametitle{Knot distance graph}
  As we've enumerated and classified every planar diagram with at most
  9 crossings, we can create a graph of the knots based on their
  distance (for $k$ up to 9):

  \begin{block}{Knot distance graph}
    Between any two vertices (knots), add an edge if there exists
    crossing switch on a planar diagram with $k$ or fewer crossings
    which toggles between the two knot types.
  \end{block}

\end{frame}



\begin{frame}
  \frametitle{Knot distance graph for 6-crossing diagrams}
  \begin{figure}
    \centering
    \includegraphics[width=\textwidth,height=.7\textheight,keepaspectratio]{6x_distances.pdf}
    \caption{Knot distance graph up to and including diagrams of 6
      crossings. \href{http://hchapman.github.io/research/6x_graph.html}{Interactive
        version here.}}
    \label{fig:6xgraph}
  \end{figure}
\end{frame}

\begin{frame}
  \frametitle{Knot distance graph for 9-crossing diagrams}
  \begin{figure}
    \centering
    \vspace{-.17\textheight}
    \includegraphics[width=\textwidth,height=\textheight,keepaspectratio]{9x_distances.pdf}
    \vspace{-.17\textheight}
    \caption{Knot distance graph up to and including diagrams of 9
      crossings. \href{http://hchapman.github.io/research/9x_graph.html}{Interactive
        version here.}}
    \label{fig:9xgraph}
  \end{figure}
\end{frame}

\begin{frame}
  \frametitle{Distance tabulations using the 9-crossing graph}
      \begin{table}
      \centering\small
      \begin{tabular}{r|ccccccccccc}
                & $0_1$ & $3_1$ & $4_1$      & $5_1$      & $5_2$ & $6_1$      & $6_2$ & $6_3$ & $3_1 \# 3_1$ & $3_1 \# 3_1^*$ \\
        \hline
        $3_1$   & 1     & 0     & 2          & 1          & 1     & 2          & 1     & 1     & 1            & 1              \\
        $3_1^*$ & 1     & 2     & 2          & 3          & 2     & 2          & 2     & 1     & 3            & 1              \\
        $4_1$   & 1     & 2     & 0          & \textcolor{red}{3} & 2     & 1          & 1     & 2     & \textcolor{red}{3}   & \textcolor{red}{3}     \\
        $5_1$   & 2     & 1     & \textcolor{red}{3} & 0          & 1     & \textcolor{red}{3} & 2     & 2     & 2            & 2              \\
        $5_1^*$ & 2     & 3     & \textcolor{red}{3} & 4          & 3     & \textcolor{red}{3} & 3     & 2     & 4            & 2              \\
        $5_2$   & 1     & 1     & 2          & 1          & 0     & 2          & 2     & 2     & 2            & 2              \\
        $5_2^*$ & 1     & 2     & 2          & 3          & 2     & 2          & 2     & 2     & 3            & 2              \\
        $6_1$   & 1     & 2     & 1          & \textcolor{red}{3} & 2     & 0          & 1     & 2     & \textcolor{red}{3}   & \textcolor{red}{3}     \\
        $6_1^*$ & 1     & 2     & 1          & \textcolor{red}{3} & 2     & 1          & 2     & 2     & \textcolor{red}{3}   & \textcolor{red}{3}     \\
      \end{tabular}
      \caption{Distances calculated from the 9-crossing distance
        graph. Cells in red are the upper bound for unknowns in Darcy's table}
      \label{fig:disttable}
    \end{table}

\end{frame}

\begin{frame}
  \frametitle{Transitions between knot types}
  We can additionally define the weight of any edge in the graph:

  \begin{block}{Edge weights}
    Let the weight of an edge $(K, L)$ between two knot types be the
    number of pairs $(D, x)$ of diagrams with knot type $K$ and
    crossings $x \in D$ so that toggling the sign of $x$ produces a
    diagram $D'$ which is of knot type $L$.
  \end{block}
\end{frame}

\begin{frame}
  \frametitle{Transition probabilities between knot types}
  \begin{figure}
    \centering
    \def\svgwidth{0.6\textheight}
    \input{chord_diagram_distances.pdf_tex}
    \caption{Chord diagram of the adjacency matrix of the knot
      distance graph. \href{http://hchapman.github.io/research/9x_chords.html}{Interactive version here.}}
    \label{fig:chorddia}
  \end{figure}
\end{frame}

\section{Additional information}
\bibliographystyle{alpha}


\end{document}

%%% Local Variables:
%%% mode: latex
%%% TeX-master: t
%%% End:
