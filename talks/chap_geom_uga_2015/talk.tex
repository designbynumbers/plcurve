\documentclass[12pt]{amsart}
\usepackage{setspace}
\usepackage{amsthm,bbm}
\usepackage[margin=1in]{geometry}

\newcommand{\A}{\mathcal{A}}
\newcommand{\B}{\mathcal{B}}
\newcommand{\C}{\mathcal{C}}

\newcommand{\Z}{\mathbb{Z}}
\newcommand{\R}{\mathbb{R}}
\newcommand{\Prb}{\mathbb{P}}

\theoremstyle{definition}
\newtheorem*{definition}{Definition}
\newtheorem*{notat}{Notation}
\newtheorem*{example}{Example}
\newtheorem*{fact}{Fact}
\newtheorem*{theorem}{Theorem}
\newtheorem*{corollary}{Corollary}
\theoremstyle{plain}

\title{Asymptotic laws for knot diagrams}
\author{Harrison Chapman}

\begin{document}
\begin{abstract}
  We consider a model of random knots akin to the one proposed by
Dunfield et.\ al.; a random knot diagram is a random immersion of the
circle into the sphere with randomly assigned crossings. By studying
diagrams as annotated planar maps, we are able to show that any given ``tangle
diagram'' almost certainly occurs many times in a random knot diagram
with sufficiently many crossings. Thus, in this model, it is
exponentially unlikely for a diagram with $n$ crossings to represent
an unknot as $n \rightarrow \infty$. This asymptotic behavior is
similar to that seen in other models of random knots such as random
lattice walks and random polygons.
\end{abstract}
\maketitle

\doublespacing

\section{Knot diagrams}

A \emph{knot} is an embedding of the circle into $\R^3$, modulo
\emph{ambient isotopy}, wherein the knot is allowed to move around in
space, but not pass through itself.

For the sake of this talk note that there is a knot which is
``trivial,'' the \emph{unknot}. The unknot class of knots contains the
closed ring-shaped loop. We'll only really care whether or not something is
\emph{knotted}, i.e., not representing the unknot.

Knots provide a model for physical polymers, so we would like to know
how to sample a knot \emph{randomly}.

\textbf{Different methods}.
\begin{enumerate}
\item \textbf{Geometric}. Random space curves, random lattice walks
\item \textbf{Combinatorial}. Random 'braid words,' the
  \emph{Petaluma} model in which a knot is associated to any given permutation.
\end{enumerate}
 [[ describe Petaluma and its results, they're cool. ]]

\vspace{2in}

There's no clear way to go from one model to another, and it is harder
to ``cross worlds;'' there is no clear connection between a geometric
model and a combinatorial model.

The idea was to study ``pictures'' of knots called \emph{knot
  diagrams}, but they're actually \textbf{tough} to study because of
their symmetries. So... we'll break the symmetry, and do some magic!

This idea comes from the study of combinatorial objects called
\emph{maps} (in the sense of ``cartography''):
\begin{definition}
  A \emph{map} $M = (V, E, F)$ is a particular embedding of a graph $G = (V, E)$ on a
  surface $\Sigma$, such that all faces are disks.

  A \emph{map isomorphism} is an orientation preserving homeomorphism of the
  surfaces which maps vertices to vertices, edges to edges, and faces to
  faces.

  A \emph{rooted map} is a map together with a choice of edge and a
  choice of a face adjacent that edge.

  Rooted map isomorphism additionally requires that the roots map to
  the roots. This ensures a rooted map has only trivial automorphism group.
\end{definition}
[[ draw maps, rooted maps, explain notation... root face is exterior ]]

\vspace{2in}

\textbf{Idea}: Knot diagrams are just a subset of spherical maps whose vertices
are colored with a choice of \emph{sign} (over-under information).

\begin{definition}
  A \emph{rooted knot diagram}, or \emph{knot diagram} (for the scope
  of this talk), $K$ is a rooted map $(V, E, F)$ of the sphere $S^2$
  together with a map $\chi: V \to \{+, -\}$ which has only one
  \emph{link component}.
\end{definition}
[[ draw what link component means.. edges across vertices.. ]]

\vspace{1in}

By then extending definitions appropriately, we can answer the
questions which follow for specific subsets of knot diagrams as well,
provided apt constructions. The biggest subset of interest is
\emph{prime knot diagrams}, which have no pair of edges which
disconnect a diagram:

\vspace{1in}

\section{A pattern theorem}

The primary thing that I wanted to ascertain in the diagram model
(rooted \emph{or unrooted}), is that, like random space curves or
random combinatorial models, the probability of being unknotted tends
to zero as the size of the object increases.

This results from a ``pattern theorem,'' which says that in a
sufficiently large object any substructure which can appear often
will. The proof of this however requires that we even know that there
\emph{are} enough knot diagrams!

\begin{theorem}[C- 2015]
  (Extension of a theorem of Bender, Gao, Richmond 1992) The set of
  rooted knot diagrams grows ``smoothly,'' i.e.,
  $\lim_{n\to\infty} k_n^{1/n}$ exists.
\end{theorem}

\begin{proof}[Key construction used in proof]
  The proof boils essentially down to providing a construction where
  \begin{enumerate}
  \item given some specific size $n$, there exists $n < m$ and
    injections $\phi_0, \phi_1$ from diagrams of size $n$ into
    diagrams of size $m$ and diagrams of size $m+1$ and
  \item there exists a ``product'' operation on diagrams constructed
    by $\phi_*$, combining two diagrams with their sizes changing
    additively, so that any diagram has at most one ``prime
    decomposition.''
  \end{enumerate}
[[ draw construction for arb knot diagrams ]]

\vspace{3in}
\end{proof}

This lets us show a ``pattern theorem,'' that shows that, in the
limit, any nice substructure appears often in a given knot
diagram. The substructure for knot diagrams are tangles;

\begin{definition}
  A \emph{$2k$-tangle} is the interior of a generic dual cycle of a
  knot diagram.

  A knot diagram contains a tangle if there is an appropriate
  homeomorphism from the surface with boundary of the tangle to the
  surface of the knot diagram
\end{definition}
[[ draw pictures, give intuition... is basically just if you cut out a
disk from a knot diagram ]]

\vspace{1in}

\newcommand{\KnotShad}{\mathcal{K}}
\newcommand{\KDefShad}{\mathcal{H}}

\begin{theorem}[C- 2015]
  (Porism of a theorem of BGR 1992) Let $\KnotShad$ be some set of
  knot diagrams, and let $P$ be a tangle (on a surface of genus
  0). Let $\KDefShad$ be the subset of diagrams in $\KnotShad$ with
  $n$ crossings that contain less than $cn$ pairwise disjoint copies
  of $P$ as a subtangle.

  Suppose there exists a method of ``attachment'' of $P$ into a knot
  diagram $D \in \KDefShad$ such that
  \begin{enumerate}
  \item for some fixed positive integer $\ell$, if $n$ is the number
    of crossings in $D$, at least $\lfloor
    n/\ell \rfloor$ possible non-conflicting places of attachment
    exist,
  \item only knot diagrams in $\KnotShad$ are produced,
  \item for any diagram produced as such we can identify the copies of
    $P$ that have been added and they are all pairwise disjoint, and
  \item given the copies that have been added, the original map and
    the associated places of attachment are uniquely determined.
  \end{enumerate}
  If $1 > c > 0$ is sufficiently small and $\KnotShad$ grows
  ``smoothly,'' then there exist constants $d < 1$ and $N > 0$ so that
  for all $n \ge N$,
  \[ \Prb(\text{a $n$-k.d. has fewer than $cn$ copies of $P$}) = \frac{\#\KDefShad_n}{\#\KnotShad_n} < d^n.\]
\end{theorem}

\begin{proof}[Viable attachments for knot diagrams]
  Again the proof relies on the construction of an attachment. The
  obvious attachment for diagrams which are allowed to not be prime is
  \emph{connect sum}, wherein a $2$-tangle is added to a diagram by a
  ``cut and splice'' procedure.
  \vspace{1in}

  For prime subclasses of diagrams, it is possible to work with an
  attachment in which a vertex of the original knot diagram is
  replaced by a $4$-tangle in an appropriate way. There are
  \emph{extra considerations for this}, namely that $P$ must be
  packaged in a ``capsid'' that eliminates the ambiguity for
  requirement (3).
  \vspace{1in}

\end{proof}

The pattern theorem proves first-off that almost every knot diagram (in the
limit) is knotted, in many different ``classes'' of knot diagrams. For
example in the case of all knot diagrams, take $P$ to be a trefoil
connect summand---so almost every knot diagram cannot be the unknot
since it has a trefoil component.

\section{Asymmetry and unrooted diagrams}

The pattern theorem also provides a result which I've found cited as
``probable fact'' in (Zinn Justin 2004, Coqueraux-Zuber 2015) but
never quite proved properly for knot diagrams:

\begin{theorem}[Richmond-Wormald 1995] (Application to certain classes
  of knot diagrams: C- 2015). Let $\KnotShad$ be one of several
  classes of rooted knot diagrams for which there is a pattern theorem
  for a tangle $P$ so that
  \begin{enumerate}
  \item $P$ has no reflective symmetry in the plane preserving its
    boundary and,
  \item For any diagram $D$ containing a copy of $P$, all diagrams
    obtainable by removing $P$ and gluing it back in are in
    $\KnotShad$.
  \end{enumerate}
  Then the proportion of $n$-crossing diagrams in $\KnotShad$ with nontrivial
  automorphisms is exponentially small.
\end{theorem}
\begin{proof}[Construction for arbitrary knot diagrams]
  \vspace{1in}

\end{proof}
\begin{corollary}
  Asymptotically, unrooted diagrams behave like rooted diagrams.
\end{corollary}
\begin{corollary}
  Unrooted diagrams have the pattern theorem
\end{corollary}
\begin{corollary}
  An unrooted diagram is almost surely knotted.
\end{corollary}

\section{Future directions}

\subsection{Random sampling}
There is a beautiful bijection between rooted 4-valent maps and
blossom trees (Schaeffer 2003?).
\vspace{2in}

Knot diagrams without crossing information look like 4-valent maps,
but not all 4-valent maps with crossing information look like knot
diagrams. Can write rejection sampler--but asymptotically most
diagrams produced this way are \emph{not knot diagrams}!

Connected to this is the problem of actual \emph{counts} for diagrams
in certain classes. This problem is \emph{hard}! (ZinnJustin,
Schaeffer, Jacobsen, Zuber, et. al. et al)

\subsection{Other classes of diagrams}
Provided constructions, this toolkit can prove asymptotics for other
classes of knot diagrams. Diagrams without disconnecting vertices?
Etc...

On that note---what about diagrams with different kinds of signs? Can
consider a ``skein'' tangency,
\vspace{1in}

Then what about skein knot diagrams (maps together with a map from
vertices into $\{+, -, s\}$)? What about other kinds of tangencies
(e.g. Jones/Khovanov type tangencies)?

\end{document}

%%% Local Variables:
%%% mode: latex
%%% TeX-master: t
%%% End:
