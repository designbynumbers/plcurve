\PassOptionsToPackage{table}{xcolor}
\documentclass[presentation,t]{beamer}
%\usetheme{Antibes}
%\usetheme{Boadilla}
\usetheme{default}
\usecolortheme{dolphin}
\useinnertheme{rectangles}
%\useoutertheme{infolines}
%\usepackage{amsmath}
\usepackage{amsthm}
\usepackage{amssymb}
\usepackage{graphicx}
\usepackage{fullpage}
\usepackage{enumerate}
\usepackage{array}
\usepackage{multicol}
%\usepackage{youngtab}
\usepackage{float}

\usepackage[utf8x]{inputenc}
\usepackage{default}
\usepackage{amssymb,amsfonts,amsbsy,amsmath}
%\usepackage{tkz-graph}
\usepackage{amsmath}
\usepackage{graphicx}

\newcommand{\mycirc}[1]
{\textcircled{\raisebox{-.8pt}{#1}}}

\newcommand{\mytwo}[2]
{\raisebox{1.1pt}{\scriptsize #1} \hspace{1pt} \raisebox{-1.1pt}{\scriptsize #2}}


%% Some useful operatorname declarations for nice texing.
\newcommand{\Aut}{\operatorname{Aut}}
\newcommand{\Inn}{\operatorname{Inn}}
\newcommand{\Ker}{\operatorname{Ker}}
\newcommand{\Stab}{\operatorname{Stab}}
\newcommand{\orb}[1]{\mathcal{O}_#1}
\newcommand{\lcm}{\operatorname{lcm}}
\newcommand{\U}{\operatorname U}
\newcommand{\cl}{\operatorname {cl}}

%% Stuff from King's .tex files
\newcommand{\untab}{\noindent \!\!\!\!\!\!\!\!\!}
\newcommand{\lra}{\longrightarrow}
\newcommand{\mb}[1]{\mathbf{#1}}
\newcommand{\gap}{\vspace{0.1in}}
\DeclareMathOperator{\grad}{grad}
\DeclareMathOperator{\curl}{curl}
\DeclareMathOperator{\GL}{GL}
\newcommand{\myline}
{\vspace{.2in}
  \begin{center}
    \rule{5in}{.7pt}
  \end{center}
  \vspace{.2in}}

\usepackage{parskip}

\setlength{\parindent}{2ex}
\setlength{\parskip}{2ex plus 1ex minus 1ex}


%%% Good old "not sure if equal"
\newcommand{\qe}{\stackrel{?}{=}}

%% Some quickies for the big groups
\newcommand{\Q}{\mathbb Q}
\newcommand{\F}{\mathbb F}
\newcommand{\Qp}{\mathbb Q^+}
\newcommand{\Qn}{\mathbb Q^-}
\newcommand{\Qs}{\mathbb Q^*}
\newcommand{\R}{\mathbb R}
\newcommand{\Rp}{\mathbb R^+}
\newcommand{\Rn}{\mathbb R^-}
\newcommand{\Rs}{\mathbb R^*}
\newcommand{\Z}{\mathbb Z}
%\newcommand{\N}{\mathbb N}
%\newcommand{\C}{\mathbb C}

\newcommand{\Af}{\mathbb A}
\newcommand{\PP}{\mathbb P}

\newcommand{\GF}{\mathbb GF}

\newcommand{\gl}{\mathfrak g}
\newcommand{\bl}{\mathfrak b}
\newcommand{\tl}{\mathfrak t}
\newcommand{\pl}{\mathfrak p}
\newcommand{\nl}{\mathfrak n}

\DeclareMathOperator{\sh}{sh}

\newcommand{\Orb}{\mathcal O}
\newcommand{\Var}{\mathcal V}

\newcommand{\Irr}{\operatorname{Irr}}

\newcommand{\Brak}{[\cdot,\cdot]}

\newcommand{\NWN}{\nl \cap {^w\nl}}
\newcommand{\closeGNWN}{\overline{G \cdot (\NWN)}}

\newcommand{\N}{\mathbb N}
\newcommand{\C}{\mathbb C}

\newcommand{\GLn}{GL_n(\C)}
\newcommand{\gln}{\mathfrak {gl}_n}

\newcommand{\rank}{\operatorname{Rank}}

\newcommand{\bracket}{[\cdot, \cdot]}

\newcommand{\LRGfxSymArrowTwo}[2]{%
  \raisebox{1.3cm}{ \centering \parbox{.6cm}{ \centering%
    {\scriptsize $#1$}\\[-.23cm]$\longrightarrow$\\[-.3cm]%
    $\longleftarrow$\\[-.23cm] {\scriptsize $#2$}%
  } }%
}

% \newcommand{\LRGfxSymArrowTwo}[2]{
%   \raisebox{1.3cm}{
%     \begin{tabular}{c}
%       \centering%
%       {\scriptsize $#1$}\\[-.23cm]
%       $\longrightarrow$\\[-.3cm]%
%       $\longleftarrow$\\[-.23cm]
%       {\scriptsize $#2$}%
%     \end{tabular}
%   }
% }

\newcommand{\LRGfxSymArrow}[1]{
  \LRGfxSymArrowTwo{#1}{#1}
}
%% Good old "not sure if equal"
\newcommand{\qe}{\stackrel{?}{=}}

\makeatletter
\DeclareRobustCommand{\em}{%
  \@nomath\em \if b\expandafter\@car\f@series\@nil
  \normalfont \else \bfseries \fi}
\makeatother
\renewcommand<>{\emph}[1]{{\em #1}}

%% Some quickies for the big groups
\newcommand{\Q}{\mathbb Q}
\newcommand{\F}{\mathbb F}
\newcommand{\Qp}{\mathbb Q^+}
\newcommand{\Qn}{\mathbb Q^-}
\newcommand{\Qs}{\mathbb Q^*}
\newcommand{\R}{\mathbb R}
\newcommand{\Rp}{\mathbb R^+}
\newcommand{\Rn}{\mathbb R^-}
\newcommand{\Rs}{\mathbb R^*}
\newcommand{\Z}{\mathbb Z}
%\newcommand{\N}{\mathbb N}
%\newcommand{\C}{\mathbb C}

\newcommand{\Af}{\mathbb A}
\newcommand{\PP}{\mathbb P}

\newcommand{\GF}{\mathbb GF}

\newcommand{\gl}{\mathfrak g}
\newcommand{\bl}{\mathfrak b}
\newcommand{\tl}{\mathfrak t}
\newcommand{\pl}{\mathfrak p}
\newcommand{\nl}{\mathfrak n}

\DeclareMathOperator{\sh}{sh}

\newcommand{\Orb}{\mathcal O}
\newcommand{\Var}{\mathcal V}

\newcommand{\Irr}{\operatorname{Irr}}

\newcommand{\Brak}{[\cdot,\cdot]}

\newcommand{\NWN}{\nl \cap {^w\nl}}
\newcommand{\closeGNWN}{\overline{G \cdot (\NWN)}}

\newcommand{\N}{\mathbb N}
\newcommand{\C}{\mathbb C}

\newcommand{\GLn}{GL_n(\C)}
\newcommand{\gln}{\mathfrak {gl}_n}

\newcommand{\rank}{\operatorname{Rank}}


\newcommand{\loopinsert}{E_1}
\newcommand{\edgedouble}{E_2}
\newcommand{\cutedgedouble}{E_3}
\newcommand{\pairinsert}{E_4}
\newcommand{\plantri}{\textit{plantri} }
\newcommand{\nauty}{\textit{nauty} }
\newcommand{\saucy}{\textit{saucy} }
\newcommand{\valgrind}{\textit{valgrind} }


\usepackage{overpic}
%\usepackage{svg}
\usepackage{graphicx}
%\usepackage{mathrsfs}
%\usepackage{setspace}
%\usepackage{showkeys}
\usepackage{amsmath}
\usepackage{hyperref}
\usepackage{tikz}
\usepackage{pgfplots}
\usepackage{pgfplotstable}
\usepackage{array}
\pgfplotsset{compat=newest}
\usetikzlibrary{positioning,arrows,knots,calc,decorations.markings}
\usepgfplotslibrary{colorbrewer}

\definecolor{beamerblue}{RGB}{234,233,243}
\definecolor{beamerviolet}{RGB}{47,23,132}
\definecolor{beamerliteviolet}{RGB}{137,127,207}

\tikzset{onslide/.code args={<#1>#2}{%
  \only<#1>{\pgfkeysalso{#2}} % \pgfkeysalso doesn't change the path
}}
\tikzset{temporal/.code args={<#1>#2#3#4}{%
  \temporal<#1>{\pgfkeysalso{#2}}{\pgfkeysalso{#3}}{\pgfkeysalso{#4}} % \pgfkeysalso doesn't change the path
}}

\tikzset{
  invisible/.style={opacity=0},
  visible on/.style={alt={#1{}{invisible}}},
  alt/.code args={<#1>#2#3}{%
    \alt<#1>{\pgfkeysalso{#2}}{\pgfkeysalso{#3}} % \pgfkeysalso doesn't change the path
  },
}

\pgfplotsset{
  /pgfplots/bar cycle list/.style={/pgfplots/cycle list={%
      {RdBu-10-1!50!black,fill=RdBu-10-1!80!white,mark=none},%
      {RdBu-10-2!50!black,fill=RdBu-10-2!80!white,mark=none},%
      {RdBu-10-3!50!black,fill=RdBu-10-3!80!white,mark=none},%
      {RdBu-10-4!50!black,fill=RdBu-10-4!80!white,mark=none},%
      {RdBu-10-5!50!black,fill=RdBu-10-5!80!white,mark=none},%
      {RdBu-10-6!50!black,fill=RdBu-10-6!80!white,mark=none},%
      {RdBu-10-7!50!black,fill=RdBu-10-7!80!white,mark=none},%
      {RdBu-10-8!50!black,fill=RdBu-10-8!80!white,mark=none},%
      {RdBu-10-9!50!black,fill=RdBu-10-9!80!white,mark=none},%
      {RdBu-10-10!50!black,fill=RdBu-10-10!80!white,mark=none},%
    }
  },
}

\pgfplotsset{
  /pgfplots/area cycle list/.style={/pgfplots/cycle list={%
      {RdBu-10-1!50!black,fill=RdBu-10-1!80!white,mark=none},%
      {RdBu-10-2!50!black,fill=RdBu-10-2!80!white,mark=none},%
      {RdBu-10-3!50!black,fill=RdBu-10-3!80!white,mark=none},%
      {RdBu-10-4!50!black,fill=RdBu-10-4!80!white,mark=none},%
      {RdBu-10-5!50!black,fill=RdBu-10-5!80!white,mark=none},%
      {RdBu-10-6!50!black,fill=RdBu-10-6!80!white,mark=none},%
      {RdBu-10-7!50!black,fill=RdBu-10-7!80!white,mark=none},%
      {RdBu-10-8!50!black,fill=RdBu-10-8!80!white,mark=none},%
      {RdBu-10-9!50!black,fill=RdBu-10-9!80!white,mark=none},%
      {RdBu-10-10!50!black,fill=RdBu-10-10!80!white,mark=none},%
    }
  },
}


% \pgfplotsset{
%   /pgfplots/bar cycle list/.style={/pgfplots/cycle list={%
%       {red!80!black,fill=red!30!white,mark=none},%
%       {orange!80!black,fill=orange!30!white,mark=none},%
%       {yellow!80!black,fill=yellow!30!white,mark=none},%
%       {lime!80!black,fill=lime!30!white,mark=none},%
%       {green!80!black,fill=green!30!white,mark=none},%
%       {teal!80!black,fill=teal!30!white,mark=none},%
%       {cyan!80!black,fill=cyan!30!white,mark=none},%
%       {blue!80!black,fill=blue!30!white,mark=none},%
%       {violet!80!black,fill=violet!30!white,mark=none},%
%       {purple!80!black,fill=purple!30!white,mark=none},%
%     }
%   },
% }


\newcommand{\Uhyp}{\mathcal U} \newcommand{\Vhyp}{\mathcal V}
\newcommand{\Rhyp}{\mathcal R}

%\allowdisplaybreaks
%\usepackage[bookmarks,bookmarksopen,bookmarksdepth=4]{hyperref}

\newcommand{\so}[1]{\mathfrak {so}(#1)}

%\newtheorem{theorem}{Theorem}[section]
%\newtheorem{lemma}[theorem]{Lemma}
%\newtheorem{definition}[theorem]{Definition}
\newtheorem{proposition}[theorem]{Proposition}
%\newtheorem{corollary}[theorem]{Corollary}

%\renewcommand*\showkeyslabelformat[1]{\normalfont\tiny\ttfamily(#1)}

\graphicspath{{../../figs/}{figs/}}

\let\oldemptyset\emptyset
\let\emptyset\varnothing

\DeclareMathOperator{\Arm}{Arm}
\DeclareMathOperator{\Pol}{Pol}
\DeclareMathOperator{\UP}{UP}
\DeclareMathOperator{\VP}{VP}
\DeclareMathOperator{\APol}{APol}
\DeclareMathOperator{\Diff}{Diff}
\DeclareMathOperator{\Sympl}{Sympl}
\DeclareMathOperator{\ev}{ev}
\DeclareMathOperator{\Ad}{Ad}
\DeclareMathOperator{\crit}{crit}
\DeclareMathOperator{\ind}{ind}
\DeclareMathOperator{\intr}{int}
\DeclareMathOperator{\Hom}{Hom}
\DeclareMathOperator{\Ext}{Ext}
\DeclareMathOperator{\codim}{codim}
\DeclareMathOperator{\Ann}{Ann}
\DeclareMathOperator{\im}{Im}
\DeclareMathOperator{\Int}{int}

%\usepackage[table]{xcolor}

\begin{document}
\rowcolors{2}{beamerblue!25}{white}
\newcommand{\Oh}[1]{\mathcal O (#1)}
\newcommand{\g}{\mathfrak g}
\newcommand{\ShSet}{\mathcal S}
\newcommand{\LnSet}{\mathcal L}
\newcommand{\sr}{/\!\!/}

\begin{frame}
   \begin{center}
    \only<1>{\includegraphics[width=4in]{../figs/random-70cr-1.png}}
    \only<2>{\includegraphics[height=3in]{../figs/random-70cr-2.png}}
    \only<3>{\includegraphics[height=3in]{../figs/random-70cr-3.png}}
   \end{center}
\end{frame}

\begin{frame}
  \vfill
  \begin{center}
    \includegraphics[width=4in]{knottypes.png}
  \end{center}
  %\vfill
\end{frame}

\begin{frame}
  \begin{figure}
    \centering
    \begin{tikzpicture}[every axis plot/.append style={line width=2pt}]
      \begin{axis}[
        %ymode=log,
        %log ticks with fixed point,
        title={(Exact) ratios of knots in $n$-crossing diagrams},
        xlabel={\# crossings in diagram},
        ylabel={ratios of knots},
        cycle list name=Set1-9,
        legend pos=outer north east,
        legend entries={$3_1$, $4_1$, $5_1$, $5_2$, $0_1$},
        xmax=9,
        ymin=0, ymax=0.15,
        ]
        \addplot table[x=n,y expr=\thisrow{3.1}*2] {knot_freq.tsv};
        \addplot table[x=n,y=4.1] {knot_freq.tsv};
        \addplot table[x=n,y expr=\thisrow{5.1}*2] {knot_freq.tsv};
        \addplot table[x=n,y=5.2] {knot_freq.tsv};
      \end{axis}
    \end{tikzpicture}
    \label{fig:kgrow1}
  \end{figure}
\end{frame}

\begin{frame}
  \begin{figure}
    \centering
    \begin{tikzpicture}[%scale=1,
      every axis plot/.append style={line width=2pt}]
      \begin{axis}[
        %ymode=log,
        %log ticks with fixed point,
        title={(Experimental) ratios of knots in $n$-crossing (rooted) diagrams},
        xlabel={\# crossings in diagram},
        ylabel={ratios of knots},
        cycle list name=Set1-9,
        legend pos=outer north east,
        legend entries={$3_1$, $4_1$, $5_1$, $5_2$, $3_1\#3_1$, $3_1\#3_1^*$},
        xmax=9,
        ymin=0, ymax=0.15,
        ]
        \addplot table[x=n,y expr=\thisrow{3.1}/\thisrow{total}] {sample.tsv};
        \addplot table[x=n,y expr=\thisrow{4.1}/\thisrow{total}] {sample.tsv};
        \addplot table[x=n,y expr=\thisrow{5.1}/\thisrow{total}] {sample.tsv};
        \addplot table[x=n,y expr=\thisrow{5.2}/\thisrow{total}] {sample.tsv};
      \end{axis}
    \end{tikzpicture}
    \label{fig:kgrow1}
  \end{figure}
\end{frame}

\begin{frame}
  \vfill
  \begin{center}
    \includegraphics[width=1.6in]{granny.png}

    \includegraphics[width=1.6in]{square.png}
    \\
    {Granny knot $3_1\#3_1$ (top) vs square knot $3_1\#3_1^*$ (bottom)}
  \end{center}
  %\vfill
\end{frame}

\begin{frame}
  \begin{figure}
    \centering
    \begin{tikzpicture}[%scale=1,
      every axis plot/.append style={line width=2pt}]
      \begin{axis}[
        %ymode=log,
        %log ticks with fixed point,
        title={Ratios of knots in $n$-crossing diagrams},
        xlabel={\# crossings in diagram},
        ylabel={ratios of knots},
        cycle list name=Set1-7,
        legend pos=outer north east,
        legend entries={$3_1$, $4_1$, $5_1$, $5_2$, $3_1\#3_1$, $3_1\#3_1^*$, $0_1$},
        xmin=0, xmax=70,
        ymin=0, ymax=1,
        ]
        \addplot table[x=n,y expr=\thisrow{3.1}/\thisrow{total}] {sample.tsv};
        \addplot table[x=n,y expr=\thisrow{4.1}/\thisrow{total}] {sample.tsv};
        \addplot table[x=n,y expr=\thisrow{5.1}/\thisrow{total}] {sample.tsv};
        \addplot table[x=n,y expr=\thisrow{5.2}/\thisrow{total}] {sample.tsv};
        \addplot table[x=n,y expr=\thisrow{3.1c3.1}/\thisrow{total}] {sample.tsv};
        \addplot table[x=n,y expr=\thisrow{3.1c3.1m}/\thisrow{total}] {sample.tsv};
        \addplot table[x=n,y expr=\thisrow{0.1}/\thisrow{total}] {sample.tsv};
      \end{axis}
    \end{tikzpicture}
    \label{fig:kgrow1}
  \end{figure}
\end{frame}

\begin{frame}
  \begin{figure}
    \centering
    \begin{tikzpicture}[%scale=1,
      every axis plot/.append style={line width=2pt}]
      \begin{axis}[
        %ymode=log,
        %log ticks with fixed point,
        title={Ratios of knots in $n$-crossing diagrams},
        xlabel={\# crossings in diagram},
        ylabel={ratios of knots},
        cycle list name=Set1-7,
        legend pos=outer north east,
        legend entries={$3_1$, $4_1$, $5_1$, $5_2$, $3_1\#3_1$, $3_1\#3_1^*$, $0_1$},
        xmin=0, xmax=70,
        ymin=0, ymax=0.1,
        ]
        \addplot table[x=n,y expr=\thisrow{3.1}/\thisrow{total}] {sample.tsv};
        \addplot table[x=n,y expr=\thisrow{4.1}/\thisrow{total}] {sample.tsv};
        \addplot table[x=n,y expr=\thisrow{5.1}/\thisrow{total}] {sample.tsv};
        \addplot table[x=n,y expr=\thisrow{5.2}/\thisrow{total}] {sample.tsv};
        \addplot table[x=n,y expr=\thisrow{3.1c3.1}/\thisrow{total}] {sample.tsv};
        \addplot table[x=n,y expr=\thisrow{3.1c3.1m}/\thisrow{total}] {sample.tsv};
        \addplot table[x=n,y expr=\thisrow{0.1}/\thisrow{total}] {sample.tsv};
      \end{axis}
    \end{tikzpicture}
    \label{fig:kgrow1}
  \end{figure}
\end{frame}

\begin{frame}
  \begin{figure}
    \centering
    \begin{tikzpicture}[%scale=1,
      every axis plot/.append style={line width=2pt}]
      \begin{axis}[
        ymode=log,
        log ticks with fixed point,
        title={Ratios of knots in $n$-crossing diagrams (log plot)},
        xlabel={\# crossings in diagram},
        ylabel={log(ratios of knots)},
        cycle list name=Set1-7,
        legend pos=outer north east,
        legend entries={$3_1$, $4_1$, $5_1$, $5_2$, $3_1\#3_1$, $3_1\#3_1^*$, $0_1$},
        xmin=0, xmax=70,
        ymin=0, ymax=1,
        ]
        \addplot table[x=n,y expr=\thisrow{3.1}/\thisrow{total}] {sample.tsv};
        \addplot table[x=n,y expr=\thisrow{4.1}/\thisrow{total}] {sample.tsv};
        \addplot table[x=n,y expr=\thisrow{5.1}/\thisrow{total}] {sample.tsv};
        \addplot table[x=n,y expr=\thisrow{5.2}/\thisrow{total}] {sample.tsv};
        \addplot table[x=n,y expr=\thisrow{3.1c3.1}/\thisrow{total}] {sample.tsv};
        \addplot table[x=n,y expr=\thisrow{3.1c3.1m}/\thisrow{total}] {sample.tsv};
        \addplot table[x=n,y expr=\thisrow{0.1}/\thisrow{total}] {sample.tsv};
      \end{axis}
    \end{tikzpicture}
    \label{fig:kgrow1}
  \end{figure}
\end{frame}

%%%%
\makeatletter
\def\pgfplots@getautoplotspec into#1{%
    \begingroup
    \let#1=\pgfutil@empty
    \pgfkeysgetvalue{/pgfplots/cycle multi list/@dim}\pgfplots@cycle@dim
    %
    \let\pgfplots@listindex=\pgfplots@numplots
    %%% Start new code
    \pgfkeysgetvalue{/pgfplots/cycle list set}\pgfplots@listindex@set
    \ifx\pgfplots@listindex@set\pgfutil@empty
    \else
        \c@pgf@counta=\pgfplots@listindex
        \c@pgf@countb=\pgfplots@listindex@set
        \advance\c@pgf@countb by -\c@pgf@counta
        \globaldefs=1\relax
        \edef\setshift{%
            \noexpand\pgfkeys{
                /pgfplots/cycle list shift=\the\c@pgf@countb,
                /pgfplots/cycle list set=
            }
        }%
        \setshift%
    \fi
    %%% End new code
    \pgfkeysgetvalue{/pgfplots/cycle list shift}\pgfplots@listindex@shift
    \ifx\pgfplots@listindex@shift\pgfutil@empty
    \else
        \c@pgf@counta=\pgfplots@listindex\relax
        \advance\c@pgf@counta by\pgfplots@listindex@shift\relax
        \ifnum\c@pgf@counta<0
            \c@pgf@counta=-\c@pgf@counta
        \fi
        \edef\pgfplots@listindex{\the\c@pgf@counta}%
    \fi
    \ifnum\pgfplots@cycle@dim>0
        % use the 'cycle multi list' feature.
        %
        % it employs a scalar -> multiindex map like
        % void fromScalar( size_t d, size_t scalar, size_t* Iout, const size_t* N )
        % {
        %   size_t ret=scalar;
        %   for( int i = d-1; i>=0; --i ) {
        %       Iout[i] = ret % N[i];
        %       ret /= N[i];
        %   }
        % }
        % to get the different indices into the cycle lists.
        %--------------------------------------------------
        \c@pgf@counta=\pgfplots@cycle@dim\relax
        \c@pgf@countb=\pgfplots@listindex\relax
        \advance\c@pgf@counta by-1
        \pgfplotsloop{%
            \ifnum\c@pgf@counta<0
                \pgfplotsloopcontinuefalse
            \else
                \pgfplotsloopcontinuetrue
            \fi
        }{%
            \pgfkeysgetvalue{/pgfplots/cycle multi list/@N\the\c@pgf@counta}\pgfplots@cycle@N
            % compute list index:
            \pgfplotsmathmodint{\c@pgf@countb}{\pgfplots@cycle@N}%
            \divide\c@pgf@countb by \pgfplots@cycle@N\relax
            %
            \expandafter\pgfplots@getautoplotspec@
                \csname pgfp@cyclist@/pgfplots/cycle multi list/@list\the\c@pgf@counta @\endcsname
                {\pgfplots@cycle@N}%
                {\pgfmathresult}%
            \t@pgfplots@toka=\expandafter{#1,}%
            \t@pgfplots@tokb=\expandafter{\pgfplotsretval}%
            \edef#1{\the\t@pgfplots@toka\the\t@pgfplots@tokb}%
            \advance\c@pgf@counta by-1
        }%
    \else
        % normal cycle list:
        \pgfplotslistsize\autoplotspeclist\to\c@pgf@countd
        \pgfplots@getautoplotspec@{\autoplotspeclist}{\c@pgf@countd}{\pgfplots@listindex}%
        \let#1=\pgfplotsretval
    \fi
    \pgfmath@smuggleone#1%
    \endgroup
}

\pgfplotsset{
    cycle list set/.initial=
}
\makeatother
%%%%


\begin{frame}
  \begin{figure}
    \centering
    \begin{tikzpicture}[%scale=1,
      every axis plot/.append style={line width=2pt}]
      \begin{axis}[
        %ymode=log,
        %log ticks with fixed point,
        stack plots=y,
        title={Ratios of knots in $n$-crossing diagrams (stacked)},
        xlabel={\# crossings in diagram},
        ylabel={ratios of knots},
        cycle list name=Set1-7,
        legend pos=outer north east,
        legend entries={$0_1$, $3_1$, $4_1$, $5_1$, $5_2$, $3_1\#3_1$, $3_1\#3_1^*$},
        xmin=0, xmax=70,
        ymin=0, ymax=1,
        ]
        \pgfplotsset{cycle list set=6}
        \addplot table[x=n,y expr=\thisrow{0.1}/\thisrow{total}] {sample.tsv};
        \pgfplotsset{cycle list set=0}
        \addplot table[x=n,y expr=\thisrow{3.1}/\thisrow{total}] {sample.tsv};
        \addplot table[x=n,y expr=\thisrow{4.1}/\thisrow{total}] {sample.tsv};
        \addplot table[x=n,y expr=\thisrow{5.1}/\thisrow{total}] {sample.tsv};
        \addplot table[x=n,y expr=\thisrow{5.2}/\thisrow{total}] {sample.tsv};
        \addplot table[x=n,y expr=\thisrow{3.1c3.1}/\thisrow{total}] {sample.tsv};
        \addplot table[x=n,y expr=\thisrow{3.1c3.1m}/\thisrow{total}] {sample.tsv};
      \end{axis}
    \end{tikzpicture}
    \label{fig:kgrow1}
  \end{figure}
\end{frame}

\begin{frame}
  \begin{figure}
    \centering
    \begin{tikzpicture}[%scale=1,
      every axis plot/.append style={line width=2pt}]
      \begin{axis}[
        %ymode=log,
        %log ticks with fixed point,
        title={Average Vassiliev-2 invariant for $n$ crossing immersions $S^1 \hookrightarrow S^3$},
        xlabel={\# crossings in diagram},
        ylabel={average $v_2^{avg}$},
        cycle list name=Set1-7,
        legend pos=outer north east,
        legend entries={$v_2^{avg}$},
        xmin=0, xmax=70,
        ymin=0, ymax=8,
        ]
        \addplot table[x=n,y expr=\thisrow{v2}] {sample.tsv};
      \end{axis}
    \end{tikzpicture}
    \label{fig:kgrow1}
  \end{figure}
\end{frame}

\end{document}

%%% Local Variables:
%%% mode: latex
%%% TeX-master: t
%%% End:
